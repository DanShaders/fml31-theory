\section{Квантовая физика}



\subsection{Энергия фотона}
Электромагнитное излучение поглощается и испускается квантами -- фотонами.\\
\df{Постоянная планка}{коэффициент, связывающий энергию кванта электромагнитого излучения с его частотой. [$\uJ \cdot \us$]}
\[ h = 6.626\:070\:15 \cdot 10^{-34} \; \uJ \cdot \us \text{ (точно)} \]
Так, фотон с частотой $\nu$ обладает энергией $h \nu$.

\subsection{Излучение абсолютно черного тела}
\law{Закон Планка}{
	Спектральная плотность энергетической светимости абсолютно черного тела температуры $T$ равна
	\[ M_{e, \nu}(\nu) = \frac {2\pi h\nu ^{3}}{c^{2}} \frac {1}{e^{h\nu /kT}-1} \]
}
\law{Закон смещения Вина}{
	Длина волны излучения АЧТ с максимальной интенсивностью равна
	\[ \lambda_{max} = \frac{b}{T} \text{,} \]
	где $b = 0.002\:898\;\um\cdot\uK$, $T$ -- температура тела.
}
\law{Закон Стефана -- Больцмана}{
	Энергетическая светимость АЧТ равен
	\[ M_e = \sigma T^4 \]
	где $\sigma$ -- постоянная Стефана -- Больцмана, $T$ -- температура тела.
}