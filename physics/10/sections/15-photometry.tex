\section{Фотометрия}



\subsection{Энергетические единицы}

\df{Энергия излучения}{энергия, переносимая излучением. [$\uJ$]}
\df{Поток излучения}{энергия, переносимая излучением за единицу времени. [$\uW$]}
\[ \Phi_e = \frac{\dlta Q_e}{\dlta t} \]
\df{Сила излучения}{мощность, переносимая излучением в некотором направлении; отношение потока излучения, распространяющегося от источника излучения внутри малого телесного угла, к этому телесному углу. [$\frac{\uW}{\usr}$]}
\[ I_e = \frac{\dlta \Phi_e}{\dlta \Omega} \]
\df{Энергетическая светимость}{мощность, переносимая излучением малого участка поверхности единичной площади. [$\frac{\uW}{\um^2}$]}
\[ M_e = \frac{\dlta \Phi_e}{\dlta S} \]
\df{Энергетическая яркость}{отношение потока излучения, испускаемого с бесконечно малой площадки источника и распространяющегося в бесконечно малом телесном угле, к площади проекции этой площадки на плоскость, перпендикулярную направлению распространения, и величине телесного угла. [$\frac{\uW}{\um^2 \cdot \usr}$]}
\[ L_e = \frac{\dlta^2 \Phi_e}{\dlta S \cos{\theta} \cdot \dlta \Omega} \]

Для некоторой энергетической фотометрической величины $X_e$ введём вспомогательную функцию $f_{X, \lambda}$, значение которой в точке $\lambda$ будет равняться значению рассматриваемой величины $X_e$, если бы излучение состояло только из волн с длинами меньшими либо равными $\lambda$, тогда спектральной плотностью $X_e$ с использованием длины волны как спектральной координаты будем называть
\[ X_{e, \lambda} = \frac{\dlta f_{X, \lambda}}{\dlta \lambda} \]

Под спектральной плотностью с использованием частоты как спектральной координаты будем понимать аналогичное значение, где вместо длины волны используется частота:
\[ X_{e, \nu} = \frac{\dlta f_{X, \nu}}{\dlta \nu} \]

\ulaw{
	Спектральные плотности для разных спектральных координат (длины волны и частоты) связаны следующим соотношением
	\[ X_{e, \lambda}(\lambda) = \frac{c}{\lambda^2} X_{e, \nu}\!\left( \frac{c}{\lambda} \right) \]
}
\begin{proof}
	\[
		X_{e, \lambda}(\lambda) =
		\frac{\dlta f_{X, \lambda}(\lambda)}{\dlta \lambda} =
		\frac{\dlta\!\left( X_e - f_{X, \nu}\!\left( \frac{c}{\nu} \right) \right)}{\dlta \lambda} =
		-\frac{\dlta f_{X, \nu}\!\left( \frac{c}{\nu} \right)}{\dlta \nu} \cdot \frac{\dlta \nu}{\dlta \lambda} =
		X_{e, \nu}\!\left( \frac{c}{\lambda} \right) \cdot -\frac{\dlta\!\left( \frac{c}{\lambda} \right)}{\dlta \lambda} =
	\]
	\[
		= X_{e, \nu}\!\left( \frac{c}{\lambda} \right) \cdot \frac{c \, \dlta \lambda}{\lambda^2 \, \dlta \lambda} =
		\frac{c}{\lambda^2} X_{e, \nu}\!\left( \frac{c}{\lambda} \right)
	\]
\end{proof}

\ulaw{
	Для изотропного источника выполняется
	\[ M_e = \pi L_e \]
}
\begin{proof}
	Рассмотрим бесконечно малую прямоугольную площадку $\dlta S$, радиус-вектор которой является нормалью к ней. Пусть её сферические координаты $\left( r,\, \theta,\, \phi \right)$.
	\[ \dlta \Omega = \frac{\dlta S}{r^2} =
		\frac{\dlta a \cdot \dlta b}{r^2} =
		\frac{r \, \dlta \theta \cdot r \sin \theta \, \dlta \phi}{r^2} =
		\dlta \theta \, \sin \theta \, \dlta \phi
	\]
	\[ M_e = \frac{\dlta \Phi_e}{\dlta S} =
		\int \frac{\dlta^2 \Phi_e}{\dlta S \cos \theta \, \dlta \Omega} \cos \theta \, \dlta \Omega =
		\int L_e \cos \theta \, \dlta \Omega =
	\]
	\[ =
		\int_{0}^{2 \pi} \int_{0}^{\frac{\pi}{2}} L_e \cos \theta \sin \theta \, \dlta \theta \dlta \phi =
		L_e \int_{0}^{2 \pi} \frac{1}{2} \dlta \phi =
		\pi L_e
	\]
\end{proof}



\subsection{Световые единицы}

\df{Световая энергия}{скалярная величина, характеризующая способность энергии, переносимой светом, вызывать у человека зрительные ощущения. [$\ulm \cdot \us$]}
\df{Световой поток}{световая энергия, переносимая за единицу времени. [$\ulm$]}
\[ \Phi_v = \frac{\dlta Q_v}{\dlta t} \]
\df{Сила света}{световой поток, переносимый излучением в некотором направлении; отношение светового потока, распространяющегося от источника излучения внутри малого телесного угла, к этому телесному углу. [$\frac{\ulm}{\usr} = \ucd$]}
\[ I_v = \frac{\dlta \Phi_v}{\dlta \Omega} \]
\df{Светимость}{световой поток, переносимый излучением малого участка поверхности единичной площади. [$\frac{\ulm}{\um^2}$]}
\[ M_v = \frac{\dlta \Phi_v}{\dlta S} \]
\df{Яркость}{отношение светового потока, испускаемого с бесконечно малой площадки источника и распространяющегося в бесконечно малом телесном угле, к площади проекции этой площадки на плоскость, перпендикулярную направлению распространения, и величине телесного угла. [$\frac{\ucd}{\um^2}$]}
\[ L_v = \frac{\dlta^2 \Phi_v}{\dlta S \cos{\theta} \cdot \dlta \Omega} \]
\df{Спектральная световая эффективность монохроматического излучения $K(\lambda)$}{физическая величина, характеризующая чувствительность человеческого глаза к воздействию на него монохроматического света. [$\frac{\ulm}{\uW}$]}
\df{Относительная спектральная световая эффективность монохроматического излучения $V(\lambda)$}{отношение световой эффективности при заданной длине волны к максимальному значению световой эффективности.}
\[ K(\lambda) = K_m V(\lambda) \]
\df{Кандела}{сила света в заданном направлении источника, испускающего монохроматическое излучение частотой $540\cdot10^{12} \; \uHz $, сила излучения которого в этом направлении составляет $\frac{1}{683}\;\frac{\uW}{\usr}$.}
\df{Люмен}{произведение канделы на стерадиан.}
Из двух последних определений следует, что $K\!\left( \frac{c}{540\cdot10^{12} \; \uHz} \right) = 683 \; \frac{\ulm}{\uW}$. Также отметим, что при дневном зрении $K_m = 683 \; \frac{\ulm}{\uW}$.\par

Для фотометрических световых единиц можно аналогично энергетическим едининцам ввести спектральные плотности, которые будем обозначать $X_{v, \lambda}$ и $X_{v, \nu}$.\par

\subsection{Соответствие световых единиц и энергетических}
Для фотометрической величины $X_v$ по определению выполняется
\[ X_v = \int X_{e, \lambda} K(\lambda) \dlta \lambda = K_m \int X_{e, \lambda} V(\lambda) \dlta \lambda \]

\newcolumntype{C}{ >{\centering\arraybackslash} m{4cm} }
\newcolumntype{D}{ >{\centering\arraybackslash} m{2.8cm}}
\newcolumntype{N}{@{}m{0pt}@{}}

\renewcommand{\arraystretch}{1.5}

\begin{center}
\begin{tabular}{|C|D|C|D|N}
\hline
\textbf{Энергетическая единица} & \textbf{Обозначение} & \textbf{Световая единица} & \textbf{Обозначение} \\
\hline
Энергия излучения & $Q_e$, $\uJ$ & Световая энергия & $Q_v$, $\ulm \cdot \us$ \\
\hline
Поток излучения & $\Phi_e$, $\uW$ & Световой поток & $\Phi_v$, $\ulm$ \\
\hline
Сила излучения & $I_e$, $\displaystyle \frac{\uW}{\usr}$ & Сила света & $I_v$, $\ucd$ & \vspace{1.5em}\phantom{} \\
\hline
Энергетическая светимость & $M_e$, $\displaystyle \frac{\uW}{\um^2}$ & Светимость & $M_v$, $\displaystyle \frac{\ulm}{\um^2}$ \\
\hline
Энергетическая яркость & $L_e$, $\displaystyle \frac{\uW}{\um^2 \cdot \usr}$ & Яркость & $L_v$, $\displaystyle \frac{\ucd}{\um^2}$ \\
\hline
\end{tabular}
\setlength\extrarowheight{0pt}
\end{center}