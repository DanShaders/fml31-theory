\section{Жидкости и газы}



\df{Парообразование}{процесс перехода тела из жидкого агрегатного состояния в газообразное.}
\df{Испарение}{процесс парообразования с поверхности жидкости при любой температуре.}
\df{Кипение}{процесс парообразования во всём объеме жидкости, который происходит, если давление насыщенных паров равно давлению жидкости в данной точке.}
\df{Температура кипения}{температура, при которой происходит кипение при данном давлении.}
TODO: факторы кипения и испарения, пар, насыщенный пар, динамическое равновесие, разные изотермы, критическая точка, абсолютная влажность, относительная влажность, [Барометрическая формула], теплота парообразования, точка росы.