\section{Распределения скоростей молекул *}



\subsection{Основные понятия}
\df{Распределение вероятностей}{закон, описывающий область значений случайной величины и соответствующие вероятности появления этих значений.}
\df{Плотность распределения случайной величины $\xi$}{один из способов описания распределения вероятности; функция такая, что если её первообразная $F(x)$, то
$\lim_{x\to+\infty}{F(x)}=1$ и $\forall x_0<x_1 : F(x_1)-F(x_0)=P(x0 \le \xi \le x1)$.}
\df{Наиболее вероятное значение случайной величины}{ордината максимального значения плотности вероятности этой случайной величины.}
\df{Дисперсия случайной величины $\sigma^2$}{среднее значение квадрата случайной величины.}
\df{Нормальное распределение $N(\mu, \sigma^2)$}{распределение, плостность вероятности которого равна
\[f(x)=\frac{1}{\sigma \sqrt{2\pi}}e^{-\frac{(x-\mu)^2}{2\sigma^2}}\]
Для распределения $N(\mu, \sigma^2)$ справедливо, что его среднее и медиана - $\mu$, а дисперсия - $\sigma^2$.}



\subsection{Распределение Максвелла}
\ulaw{
	Плотность распределение скоростей по одной из осей равна
	\[ f(v) = \sqrt{\frac{m_0}{2 \pi k T}} e^{-\frac{m_0 v^2}{2 k T}} \]
}
\begin{proof}
	Т. к. средняя скорость по одной из осей -- это величина, являющаяся средним арифметическим большого количества мало зависимых величин, распределённых примерно одинакого, то по центральной предельной теореме следует, что распределение скоростей стемится к нормальному.\par
	$\overline{v_x} = 0$, а $\overline{v_x^2} = \frac{p}{m_0 n} = \frac{k T}{m_0}$, значит, что распределение скоростей равно $N(0, \frac{k T}{m_0})$. Значит плотность вероятности равна:
	\[ f(v) = \sqrt{\frac{m_0}{2\pi k T}} e^{-\frac{m_0 v^2}{2 k T}} \]
\end{proof}

\law{Распределение Максвелла}{
	Плотность распределения скоростей молекул в идеальном газе
	\[ f(v) = 4 \pi (\frac{m_0}{2\pi k T})^{\frac{3}{2}} e^{-\frac{m_0 v^2}{2 k T}} v^2 \]
}
\begin{proof}
	TODO
\end{proof}

\ulaw{
	Наиболее вероятная скорость молекулы
	\[ v_{\text{н. в.}}=\sqrt{\frac{2kT}{m_0}}=\sqrt{\frac{2RT}{M}} \]
}
\begin{proof}
	Возьмём производную распределения Максвелла и приравняем её нулю для получения ординаты экстремима:
	\[\frac{\text{d}f}{\text{d}v}=8\pi(\frac{m_0}{2\pi k T})^{\frac{3}{2}}ve^{-\frac{m_0v^2}{2 k T}}(-\frac{m_0 v^2}{2kT}+1)\]
	\[\frac{\text{d}f}{\text{d}v}=0\]
	\[v_{\text{н. в.}}=\sqrt{\frac{2kT}{m_0}}=\sqrt{\frac{2RT}{M}}\]
\end{proof}

Утверждение без доказательства для средней скорости:\par
\[\overline{v}=\sqrt{\frac{8kT}{\pi m_0}}=\sqrt{\frac{8RT}{\pi M}}\]