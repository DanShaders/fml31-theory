\section{Распределения скоростей молекул *}



\subsection{Основные понятия}
\df{Распределение вероятностей}{закон, описывающий область значений случайной величины и соответствующие вероятности появления этих значений.}
\df{Плотность распределения случайной величины $\xi$}{один из способов описания распределения вероятности; функция такая, что если её первообразная $F(x)$, то
$\lim_{x\to+\infty}{F(x)}=1$ и $\forall x_0<x_1 : F(x_1)-F(x_0)=P(x0 \le \xi \le x1)$.}
\df{Наиболее вероятное значение случайной величины}{ордината максимального значения плотности вероятности этой случайной величины.}
\df{Дисперсия случайной величины $\sigma^2$}{среднее значение квадрата случайной величины.}
\df{Нормальное распределение $N(\mu, \sigma^2)$}{распределение, плотность вероятности которого равна
\[f(x)=\frac{1}{\sigma \sqrt{2\pi}}e^{-\frac{(x-\mu)^2}{2\sigma^2}}\]
Для распределения $N(\mu, \sigma^2)$ справедливо, что его среднее и медиана - $\mu$, а дисперсия - $\sigma^2$.}



\subsection{Распределение Максвелла}
\ulaw{
	Плотность распределение скоростей по одной из осей равна
	\[ \varphi(v) = \sqrt{\frac{m_0}{2 \pi k T}} e^{-\frac{m_0 v^2}{2 k T}} \]
}
\begin{proof}
	Т. к. средняя скорость по одной из осей -- это величина, являющаяся средним арифметическим большого количества мало зависимых величин, распределённых примерно одинакого, то по центральной предельной теореме следует, что распределение скоростей стемится к нормальному.\par
	$\overline{v_x} = 0$, а $\overline{v_x^2} = \frac{p}{m_0 n} = \frac{k T}{m_0}$, значит, что распределение скоростей равно $N(0, \frac{k T}{m_0})$. Значит плотность вероятности равна:
	\[ \varphi(v) = \sqrt{\frac{m_0}{2\pi k T}} e^{-\frac{m_0 v^2}{2 k T}} \]
\end{proof}

\law{Распределение Максвелла}{
	Плотность распределения скоростей молекул в идеальном газе
	\[ f(v) = 4 \pi (\frac{m_0}{2\pi k T})^{\frac{3}{2}} e^{-\frac{m_0 v^2}{2 k T}} v^2 \]
}
\begin{proof}
	Предположим, что проекции скорости молекулы на оси независимы, тогда
	\[ \dlta f(v_x, v_y, v_z) = \varphi(v_x) \varphi(v_y) \varphi(v_z) \dlta v_x \dlta v_y \dlta v_z \]
	\[ \dlta f(v_x, v_y, v_z) = (\frac{m_0}{2 \pi k T})^{\frac{3}{2}} e^{-\frac{m_0}{2 k T}(v_x^2+v_y^2+v_z^2)} \dlta v_x \dlta v_y \dlta v_z \]
	\[ \dlta f(v) = (\frac{m_0}{2 \pi k T})^{\frac{3}{2}} e^{-\frac{m_0 v^2}{2 k T}} \dlta v_x \dlta v_y \dlta v_z \]
	Все точки с модулем скорости от $v$ до $v + \dlta v$ находятся вне шара радиусом $v$ и внутри шара радиусом $v + \dlta v$. Соответственно, объём этого пространства -- $4 \pi v^2 \dlta v$ или с другой стороны -- $\dlta v_x \dlta v_y \dlta v_z$.
	\[ \dlta f(v) = (\frac{m_0}{2 \pi k T})^{\frac{3}{2}} e^{-\frac{m_0 v^2}{2 k T}} \cdot 4 \pi v^2 \dlta v \]
	Интегрируя, получаем
	\[ f(v) = 4 \pi (\frac{m_0}{2\pi k T})^{\frac{3}{2}} e^{-\frac{m_0 v^2}{2 k T}} v^2 \]
\end{proof}

\ulaw{
	Наиболее вероятная скорость молекулы
	\[ v_{\text{н. в.}}=\sqrt{\frac{2kT}{m_0}}=\sqrt{\frac{2RT}{M}} \]
}
\begin{proof}
	Возьмём производную распределения Максвелла и приравняем её нулю для получения ординаты экстремиума:
	\[\dlta f = 8\pi(\frac{m_0}{2\pi k T})^{\frac{3}{2}}ve^{-\frac{m_0v^2}{2 k T}}(-\frac{m_0 v^2}{2kT}+1) \dlta v\]
	\[\dlta f = 0\]
	\[v_{\text{н. в.}}=\sqrt{\frac{2kT}{m_0}}=\sqrt{\frac{2RT}{M}}\]
\end{proof}

\ulaw{
	Средняя скорость молекулы
	\[ \overline{v}=\sqrt{\frac{8kT}{\pi m_0}}=\sqrt{\frac{8RT}{\pi M}} \]
}
\begin{proof}
	Пусть $b = \frac{m_0}{2 \pi k T}$, тогда
	\[ \overline{v} = \int_{0}^{+\infty} v f(v) \,\dlta v =
	\int_{0}^{+\infty} 4 \pi (\frac{m_0}{2\pi k T})^{\frac{3}{2}} e^{-\frac{m_0 v^2}{2 k T}} v^3 \;\dlta v =
	4 \pi^{-\frac{1}{2}} b^{\frac{3}{2}} \int_{0}^{+\infty} e^{-b v^2} v^3 \;\dlta v = (1) \]
	$t = b v^2$, тогда $\dlta t = 2 b v \;\dlta v $
	\[ (1) = 4 \pi^{-\frac{1}{2}} b^{\frac{3}{2}} \int_{0}^{+\infty} e^{-t} \frac{t}{b} \frac{\dlta t}{2 b} = 
	2 (\pi b)^{-\frac{1}{2}} \int_{0}^{+\infty} e^{-t} t \;\dlta t = (2) \]
	Для последнего интеграла воспользуемся интегрированием по частям, $\int f \;\dlta g = f g - \int g \;\dlta f$, где $f = t$, $\dlta g = e^{-t} \;\dlta t$, $\dlta f = \dlta t$, $g = -e^{-t}$:
	\[ \int_{0}^{+\infty} e^{-t} t \;\dlta t = (-e^{-t} x)\bigg|_{0}^{+\infty} - \int_{0}^{+\infty} -e^{-t} \;\dlta t = 
	0 + (-e^{-t})\bigg|_{0}^{+\infty} = 1 \]
	\[ (2) = 2 (\pi b)^{-\frac{1}{2}} = 2 \sqrt{\frac{2 k T}{\pi m_0}} = \sqrt{\frac{8 k T}{\pi m_0}} = \sqrt{\frac{8RT}{\pi M}} \]
\end{proof}

