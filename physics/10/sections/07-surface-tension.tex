\section{Поверхностное натяжение}


\subsection{Сила поверхностного натяжения}
\df{Поверхностная энергия}{избыточная потенциальная энергия, которой обладают молекулы на поверхности жидкости вследствие статистически меньшего расстояния до соседних молекул.}
\df{Коэффициент поверхностного натяжения}{работа внешних сил, необходимая для изменения площади поверхности жидкости на единицу площади. [$\frac{\uJ}{\um^2}$, $\frac{\uN}{\um}$]}
\[ \sigma = \frac{A_\text{вн}}{\Delta S} \]
Для $T = T_\text{кр}$ $\sigma = 0$, т. к. нет различия между жидкостью и газом. Жидкость стремится принять такую форму, чтобы минимизировать свою потенциальную энергию. Например, при отсутствии внешних сил эта форма -- шар.\par
\df{Сила поверхностного натяжения}{сила, действующая по касательной к поверхности жидкости, стремящаяся минимизировать потенциальную энергию тела.}
\df{Поверхностное натяжение}{явление возникновения силы поверхностного натяжения.}

\ulaw{
	Для каждой точки контура на поверхности жидкости сила поверхностного натяжения направлена перпендикулярно касательной к контуру в плоскости контура, а сумма модулей сил поверхностного натяжения для контура длины $l$ равна
	\[ F = \sigma l \]
}
\begin{proof}
	TODO. Нужно доказательство
\end{proof}

\law{Формула Лапласа}{
	Добавочное давление в точке поверхности равно
	\[\Delta p = \sigma\left(\frac{1}{R_1} + \frac{1}{R_2} \right) \text{, где } R_1 \text{ и } R_2 \text{ -- радиусы главных кривизн.} \]
	Радиус кривизны берётся с плюсом, если жидкость выгнута наружу в окружающую среду; иначе, если вогнута внутрь жидкости, с минусом.
}
\begin{proof}
	Рассмотрим криволинейный прямоугольник окрестности некоторой точки поверхности жидкости. Обозначим его длины сторон за $\dlta l_1$ и $\dlta l_2$, а радиус кривизны сторон за $R_1$ и $R_2$.
	\[\dlta l_1 = \alpha_1 R_1;\;\dlta l_2 = \alpha_2 R_2\]
	Восстановим нормаль к поверхности в данной точке и найдём проекцию сил натяжения на ось, сонаправленную с нормалью:
	\[ F_{n2} = -2 F_2 \cdot \sin{\frac{\alpha_1}{2}} = - 2 \sigma \dlta l_2 \frac{\alpha_1}{2} = -\sigma R_2 \alpha_1 \alpha_2 \]
	\[ F_{n1} = -2 F_1 \cdot \sin{\frac{\alpha_2}{2}} = - 2 \sigma \dlta l_1 \frac{\alpha_2}{2} = -\sigma R_1 \alpha_1 \alpha_2 \]
	\[ F = -(F_{n_1} + F_{n_2}) \]
	\[ \Delta p \cdot \dlta l_1 \cdot \dlta l_2 = \sigma \alpha_1 \alpha_2 (R_1 + R_2) \]
	\[ \Delta p \cdot \alpha_1 \cdot \alpha_2 \cdot R_1 \cdot R_2 = \sigma \alpha_1 \alpha_2 (R_1 + R_2) \]
	\[ \Delta p \cdot R_1 \cdot R_2 = \sigma (R_1 + R_2) \]
	\[ \Delta p = \sigma \frac{R_1 + R_2}{R_1 R_2} \]
	\[ \Delta p = \sigma (\frac{1}{R_1} + \frac{1}{R_2}) \]
\end{proof}



\subsection{Явления на границах раздела}
\df{Смачивание}{явление, при котором силы взаимодействия между жидкостью и твёрдой поверхностью больше, чем силы взаимодействия между частицами самой жидкости.}
На границе раздела двух сред есть свой коэффициент поверхностного натяжения $\sigma_{ij}$.\par
В случае состояния равновесия силы приложенные к точке, где встречаются рассматриваемые среды, должны компенсироваться; если такое невозможно, то состояния равновесия не будет. Например, если для двух жидкостей и газа $\sigma_{13} > \sigma_{23} + \sigma_{12}$, то жидкость 2 растечётся мономолекулярным слоем, т. е. жидкость 1 полностью смачивается жидкостью 2.\par
\df{Краевой угол смачивания}{угол, который образуется между касательной, проведённой к поверхности фазы жидкость-газ и твёрдой поверхностью с вершиной, располагающейся в точке контакта трёх фаз, и условно измеряемый всегда внутрь жидкой фазы.}
\df{Мениск}{изогнутая поверхность жидкости.}
\df{Капиллярные явления}{подъём или опускание жидкости в узких трубках по сравнению с уровнем жидкости в широких трубах.}

\ulaw{
	Косинус краевого угла смачивания равен
	\[ \cos{\theta} = \frac{\sigma_{2\text{т}} - \sigma_{1\text{т}}}{\sigma_{12}} \]
}
\begin{proof}
	Для твердого тела и двух других несмешиваемых сред в состоянии равновесия проекция силы на касательную к поверхности твердого тела обращается в 0. Соответственно,
	\[ \sigma_{1\text{т}} \cdot \dlta l + \sigma_{12} \cos{\theta} \cdot \dlta l = \sigma_{2\text{т}} \cdot \dlta l \]
	\[ \cos{\theta} = \frac{\sigma_{2\text{т}} - \sigma_{1\text{т}}}{\sigma_{12}} \]
\end{proof}

\ulaw{
	В капилляре радиусом $r$, для которого угол краевого смачивания $\theta$, жидкость поднимется (опустится) на высоту
	\[ h = \frac{2 \sigma \cos{\theta}}{\rho gr} \]
}
\begin{proof}
	Радиус кривизны жидкости в капилляре $\displaystyle R = \frac{r}{\cos{\theta}}$
	\[ \Delta p + \rho gh = 0 \]
	\[ \frac{2 \sigma}{R} = \rho gh \]
	\[ \frac{2 \sigma \cos{\theta}}{r} = \rho gh \]
	\[ h = \frac{2 \sigma \cos{\theta}}{\rho gr} \]
\end{proof}
Заметим, что из этой теоремы очевидно, что в капиллярах, смачиваемых жидкостью, она поднимается, а в несмачиваемых -- опускается. 