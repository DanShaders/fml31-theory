\section{Основы МКТ}

\subsection{Количество вещества и молекулярные массы}
\textbf{Относительная молекулярная масса} - отношение массы молекулы данного вещества к $\frac{1}{12}$ массы атома \ce{^{12}_{6}C}. [о. е. м., безразмерная]\par
\[M_r=\frac{m_0}{\frac{1}{12}m_{0\text{С}}}\]
\textbf{Абсолютная масса} - масса частицы данного вещества. [а. е. м., кг]\par
\textbf{Атомная единица массы} - единица измерения массы, равная $\frac{1}{12}$ массы атома \ce{^{12}_{6}C}.\par

\textbf{Моль} - количество вещества, содержащееся в теле с числом молекул, численно равным постоянной Авогадро.\par
\textbf{Постоянная Авогадро} - постоянная, равная $N_A=6.02214076\cdot10^{23}\text{ моль}^{-1}$ и приближенно равная количеству атомов в 12 граммах \ce{^{12}_{6}C}.\par
\textbf{Количество вещества} - мера количества молекул в веществе. [моль]\par
\[\nu=\frac{N}{N_A}\]

\textbf{Молярная масса} - масса вещества, взятого в количестве одного моля. [$\frac{\text{кг}}{\text{моль}}$]\par
\[M = m_0 N_A = 10^{-3}M_r\,\textstyle \frac{\text{кг}}{\text{моль}}\]



\subsection{Взаимодействие молекул}
\textbf{Броуновское движение} - беспорядочное движение взвешенной частицы в жидкости или газе.\par
\textbf{Тепловое движение} - беспорядчное движение молекул в теле.\par
\textbf{Сила, возникающая между двумя молекулами:}\par
\[F(r)=-\frac{a}{r^7}+\frac{b}{r^{13}}\]
\textbf{Потенциальная энергия взаимодействия молекул:}\par
\[E_{\text{п}}(r) = -\int F(r) dr = -\frac{7a}{r^8}+\frac{13b}{r^{14}}+c = -\frac{7a}{r^8}+\frac{13b}{r^{14}}\]
(Константа $c$ равна $0$, т. к. по соглашению, $\displaystyle \lim_{r\to\infty}{E_{\text{п}}(r)}=0$)\par
\textbf{Связанное состояние молекул} - состояние, при котором две молекулы совершают колебательные движения.\par