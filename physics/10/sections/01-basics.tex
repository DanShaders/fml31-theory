\section{Основы МКТ}

\subsection{Количество вещества и молекулярные массы}
\df{Относительная молекулярная масса}{отношение массы молекулы данного вещества к $\frac{1}{12}$ массы атома \ce{^{12}_{6}C}. [о. е. м., безразмерная]}
\[M_r=\frac{m_0}{\frac{1}{12}m_{0\text{С}}}\]
\df{Абсолютная масса}{масса частицы данного вещества. [а. е. м., кг]}
\df{Атомная единица массы}{единица измерения массы, равная $\frac{1}{12}$ массы атома \ce{^{12}_{6}C}.}

\df{Моль}{количество вещества, содержащееся в теле с числом молекул, численно равным постоянной Авогадро.}
\df{Постоянная Авогадро}{постоянная, равная $N_A=6.02214076\cdot10^{23}\text{ моль}^{-1}$ и приближенно равная количеству атомов в 12 граммах \ce{^{12}_{6}C}.}
\df{Количество вещества}{мера количества молекул в веществе. [моль]}
\[\nu=\frac{N}{N_A}\]

\df{Молярная масса}{масса вещества, взятого в количестве одного моля. [$\frac{\text{кг}}{\text{моль}}$]}
\[M = m_0 N_A = 10^{-3}M_r\,\textstyle \frac{\text{кг}}{\text{моль}}\]



\subsection{Взаимодействие молекул}
\df{Броуновское движение}{беспорядочное движение взвешенной частицы в жидкости или газе.}
\df{Тепловое движение}{беспорядчное движение молекул в теле.}
\df{Связанное состояние молекул}{состояние, при котором две молекулы совершают колебательные движения.}
\law{Сила, возникающая между двумя молекулами (утв. без доказательства)}{
	\[F(r)=-\frac{a}{r^7}+\frac{b}{r^{13}}\]
}
\textbf{Потенциальная энергия взаимодействия молекул:}
\[E_{\text{п}}(r) = -\int F(r) dr = -\frac{7a}{r^8}+\frac{13b}{r^{14}}+c = -\frac{7a}{r^8}+\frac{13b}{r^{14}}\]
(Константа $c$ равна $0$, т. к. по соглашению, $\displaystyle \lim_{r\to\infty}{E_{\text{п}}(r)}=0$)\par