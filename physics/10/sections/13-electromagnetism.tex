\section{Электромагнетизм}



\subsection{Электромагнитная индукция}

\df{Поток вектора магнитной индукции для контура}{поток вектора магнитной индукции через любую поверность, натянутую на этот контур. Так как для любой замкнутой поверности $\Phi_B = 0$, то этот поток не зависит от выбранной поверности.}

\begin{wrapfigure}{R}{0.4\textwidth}
	\def\svgwidth{0.4\textwidth}
	\input{images/orientation.pdf_tex}
	\caption{Выбор направления обхода}
	\label{fig:orientation}
\end{wrapfigure}

При определении магнитного потока через поверхность, её требуется ориентировать, после этого ориентируем и контур согласно правилу буравчика (\ref{fig:orientation}).

\df{Собственный магнитный поток контура}{поток, вызываемый током в самом контуре.}
\df{Индуктивность контура}{отношение собственного магнитного потока контура к току, протекающему в нём. [$\uH = \frac{\uV \cdot \us}{\uA}$]}
\[ \Phi_\text{с} = LI \]

\law{Закон электромагнитной индукции}{
	\[ \varepsilon = -\frac{\dlta \Phi}{\dlta t} \text{,} \]
	где $\frac{\dlta \Phi}{\dlta t}$ называют скоростью изменения потока.
}

\df{Самоиндукция}{явление возникновения ЭДС в контуре при изменении силы тока или индуктивности.}

\[ \varepsilon_\text{с} = -\frac{\dlta \Phi_\text{с}}{\dlta t} = -L \frac{\dlta I}{\dlta t} - \frac{\dlta L}{\dlta t} I \]

\begin{wrapfigure}{R}{0.4\textwidth}
	\def\svgwidth{0.4\textwidth}
	\input{images/example-lenz.pdf_tex}
	\caption{Пример применения правила Ленца}
	\label{fig:example-lenz}
\end{wrapfigure}

\mbox{
	\begin{minipage}{0.6\textwidth - \columnsep}
		\law{Правило Ленца}{
			\begin{enumerate}
				\item Индукционный ток направлен так, чтобы своим магнитным полем противодействовать изменению магнитного потока, которым он вызван.
				\item Направление наведённой ЭДС всегда таково, что она пытается препятствовать причине, её вызывающей.
			\end{enumerate}
		}
	\end{minipage}
}

Рассмотрим пример (\ref{fig:example-lenz}). При удалении магнита, магнитный поток для контура уменьшается, тогда (используя первую формулировку) вектор магнитной индукции тока в контуре будет направлен направо $\thus$ ток течёт против часовой стрелки. (Используя вторую формулировку) Так как причина изменения потока -- удаление магнита, то магнитное поле будет притягивать магнит к кольцу.\par

\law{Закон электромагнитной индукции в формулировке Фарадея}{
	Для контура выполняется
	\[ q = \frac{\Phi_0 - \Phi_1}{R} \text{,} \]
	где $q$ -- ток, прошедший из-за изменения магнитного поля, $R$ -- сопротивление контура.
}

\begin{proof}
	\[ -\frac{\dlta \Phi}{\dlta t} = \varepsilon = IR = \frac{\dlta q}{\dlta t}R \]
	\[ -{\dlta \Phi} = {\dlta q}R \]
	\[ \int -{\dlta \Phi} = \int {\dlta q}R \]
	\[ q = \frac{\Phi_0 - \Phi_1}{R} \]
\end{proof}



\subsection{Индукция в движущемся проводнике}

\begin{wrapfigure}{R}{0.2\textwidth}
	\def\svgwidth{0.2\textwidth}
	\input{images/moving-conductor.pdf_tex}
	\caption{Проводник}
	\label{fig:moving-conductor}
\end{wrapfigure}

\mbox{
	\begin{minipage}{0.8\textwidth - \columnsep}
	\ulaw{
		Для проводника, движущегося в магнитном поле справдливо
		\[ \varepsilon = \int (\vec{v} \times \vec{B}) \cdot \dlta \vec{l} \]
	}
	\end{minipage}
}
\begin{proof}
	Пусть есть проводник, который движется в магнитном поле. Тогда по определению ЭДС, 
	\[ \varepsilon_{AB} = \frac{\int A_\text{ст}}{q} = \frac{\int \vec{F} \cdot \dlta \vec{l}}{q} = \frac{\int q ((\vec{u} + \vec{v}) \times \vec{B}) \cdot \dlta \vec{l}}{q} \text{,} \]
	где $\vec{u}$ -- скорость пробного заряда относительно проводника.
	\[ \varepsilon_{AB} = \int \left((\vec{v} \times \vec{B}) \cdot \dlta \vec{l} + (\vec{u} \times \vec{B}) \cdot \dlta \vec{l}\right) \]
	\[ \varepsilon_{AB} = \int \left((\vec{v} \times \vec{B}) \cdot \dlta \vec{l} - \vec{B} \cdot (\vec{u} \times \dlta \vec{l})\right) \]
	\[ \varepsilon_{AB} = \int \left((\vec{v} \times \vec{B}) \cdot \dlta \vec{l} - \vec{B} \cdot \vec{0} \right) = \int (\vec{v} \times \vec{B}) \cdot \dlta \vec{l} \]
\end{proof}

Если проводник движется поступательно в однородном магнитном поле, то последняя формула превщается в
\[ \varepsilon_{AB} = (\vec{v} \times \vec{B}) \cdot \vec{l} \]

Ещё более частные случаи этой формулы. Если проводник прямой и перпендикулярен плоскости, в которой содержатся $\vec{v}$ и $\vec{B}$, а между последними угол $\alpha$, то
\[ \abs{\varepsilon_{AB}} = v B l \cdot \sin{\alpha} \]
Если проводник не перпендикулярен, но прямой и находится под углом $\beta$ к плоскости, то
\[ \abs{\varepsilon_{AB}} = v B l \cdot \sin{\alpha} \cdot \cos{\beta} = v B l_\text{экв.} \cdot \sin{\alpha} \]



\subsection{Переменное магнитное поле}

Переменное магнитное поле создаёт вихревое электрическое поле.\par

\ulaw{
	Для осесимметричного магнитного поля
	\[ E_\text{вихр.} = \frac{\dlta \Phi}{2 \pi r \dlta t} \text{,} \]
	где $E_\text{вихр.}$ -- напряжённость вихревого электрического поля, $r$ -- расстояние от центра симметрии.
}
\begin{proof}
	Рассмотрим кольцо радиуса $r$ в осесимметричном магнитном поле, причём оси симметрии кольца и поля совпадают. Тогда,
	\[ \varepsilon = -\frac{\dlta \Phi}{\dlta t} \]
	\[ \dlta \varepsilon_i = \frac{q E_{\text{вихр., }i} \cdot \dlta l_i}{q} = E_{\text{вихр., }i} \cdot \dlta l_i \]
	Т. к. поле симметрично, то все $E_{\text{вихр., }i}$ равны.
	\[ \varepsilon = \int E_{\text{вихр., }i} \cdot \dlta l_i = E_\text{вихр.} \cdot 2 \pi r \]
	\[ E_\text{вихр.} = \frac{\dlta \Phi}{2 \pi r \dlta t} \]
\end{proof}



\subsection{Ток в контурах с индуктивностью}
\ulaw{
	Пусть есть катушка индуктивностью $L$ и сопротивлением $R$, тогда при замыкании неё на источник с ЭДС $\varepsilon$ сила тока в цепи будет подчиняться закону
	\[ I = \frac{\varepsilon}{R} \left( 1 - e^{-\frac{tR}{L}} \right) \]
}
\begin{proof}
По второму закону Кирхгофа,
\[ \varepsilon + \varepsilon_{SI} = IR \]
\[ \varepsilon - L \frac{\dlta I}{\dlta t} = IR \]
\[ \dlta t = \frac{L \dlta I}{\varepsilon - IR} \]
\[ \int \dlta t = L \int \frac{\dlta I}{\varepsilon - IR} \]
\[ u = \varepsilon - IR \text{;} \quad \dlta u = -R \dlta I \text{;} \quad \dlta I = -\frac{\dlta u}{R} \]
\[ t = -L \int \frac{\frac{\dlta u}{R}}{u} \]
\[ t = -\frac{L}{R} \ln{u} + c \]
\[ t = -\frac{L}{R} \ln{(\varepsilon - IR)} + c \]
\[ I = \frac{\varepsilon - e^{\frac{(-t+c)R}{L}}}{R} \]
\[ I = \frac{\varepsilon}{R} \left( 1 - c_2 \cdot e^{\frac{-tR}{L}} \right) \]
В момент времени $t = 0 \; \us$, $I = 0 \; \uA$:
\[ 0 = \frac{\varepsilon}{R} \left( 1 - c_2 \cdot 1 \right) \]
\[ c_2 = 1 \]
\[ I = \frac{\varepsilon}{R} \left( 1 - e^{\frac{-tR}{L}} \right)\]
\end{proof}