\section{Магнитное поле}



\subsection{Основные понятия}
\df{Магнитное поле}{вид материи, который возникает из-за движения заряженных частиц и который обнаруживается по действию на другие движущиеся заряженные частицы, в том числе токи и тела, обладающие магнитным моментом.}
\df{Сила Лоренца}{сила, действующая на движущийся заряд со стороны магнитного поля.}
\df{Вектор магнитной индукции}{силовая характеристика магнитного поля такая, что сила Лоренца со стороны этого поля, действующая на заряд $q$, движущийся со скоростью $\vec{v}$ в данной точке равна}
\[ \vec{F_\text{л}} = q [ \vec{v} \times \vec{B} ] \]
Из определения следует, что единица измерения магнитной индукции -- $\frac{\uN}{\um \cdot \uA}$.\par

Установлено, что сила Лоренца гироскопическая (её работа всегда равна 0) $\thus$ она всегда лежит в плоскости, перпендикулярной вектору скорости частицы $\thus$ такой вектор $\vec{B}$ всегда найдётся.\par

\df{Магнитный момент}{векторная физическая величина, характеризующая магнитные свойства вещества. [$\uA \cdot \um^2$]}

Для плоского контура площадью $S$ с током $I$, вектор нормали для плоскости которого $\vec{n}$ (направление выбирается так, чтобы $\vec{B}_\text{собств.} \upuparrows \vec{n}$), магнитный момент вычисляется как:
\[ \vec{p}_m = I \cdot S \cdot \vec{n} \]

Элементарный контур с током может служить индикатором внешнего магнитного поля. Если размеры контура пренебрежимо малы, то он разворачивается так, чтобы его магнитный момент был сонаправлен с вектором магнитной индуции в данной точке.\par

\df{Сила Ампера}{сила, действующая со стороны магнитного поля на элемент тока.}
\ulaw{
	\[ \dlta \vec{F}_\text{а} = [ I \dlta \vec{l} \times \vec{B}] \]
}
\begin{proof}
	Сила Ампера -- сумма сил Лоренца для всех частиц тока:
	\[ \vec{F_\text{л}} = \dlta q [ \vec{v} \times \vec{B} ] \]
	\[ \vec{F_\text{л}} = [ \dlta q \cdot \vec{v} \times \vec{B} ] \]
	\[ \int \vec{F_\text{л}} = [ \left( \int \dlta q \cdot \vec{v} \right) \times \vec{B} ] \]
	\[ \dlta \vec{F_\text{а}} = [ \left( \int \frac{\dlta q \cdot \dlta \vec{l}}{\dlta t} \right) \times \vec{B} ] \]
	\[ \dlta \vec{F_\text{а}} = [ \left( \int \frac{\dlta q}{\dlta t} \right) \dlta \vec{l} \times \vec{B} ] \]
	\[ \dlta \vec{F_\text{а}} = [ I \dlta \vec{l} \times \vec{B} ] \]
\end{proof}

\law{Закон Био -- Савара (эксперимент)}{
	Вектор магнитной индукции, создаваемый элементом тока $d \vec{l}$, равен
	\[ \dlta \vec{B} = \frac{\mu_0}{4 \pi} \frac{[I \dlta \vec{l} \times \vec{r}]}{r^3} = k_1 \frac{[I \dlta \vec{l} \times \vec{r}]}{r^3} \]
	Где $k_1$ -- коэффициент пропорциональности; $\mu_0$ -- магнитная постоянная, $I$ -- ток, текущий по элементу; $\vec{r}$ -- вектор, началом которого является положение проводника, а концом -- точка, в которой определяется индукция.
	\[ \mu_0 = 1.256\:637\:062\:12(19) \cdot 10^{-6} {\textstyle \frac{\uN}{\um^2}} \text{; } \;
	k_1 = {\displaystyle \frac{\mu_0}{4 \pi}} \approx 10^{-7} {\textstyle \frac{\uN}{\um^2}} \]
}

\ulaw{
	Поток магнитного поля (вектора магнитной индукции) через замкнутую поверхность равен
	\[ \Phi_B = 0 \]
}
\begin{proof}
	TODO?
\end{proof}

\law{Теорема о циркуляции}{
	Циркуляция вектора магнитной индукции по замкнутому контуру равна алгебраической сумме токов, охваченных произвольной поверхностью, натянутой на этот контур, умноженной на магнитную постоянную.
	\[ \oint \vec{B} \dlta \vec{l} = \mu_0 \sum I_i \]
}
\begin{proof}
	TODO?
\end{proof}

\ulaw{
	Индукция магнитного поля бесконечного соленоида на его оси симметрии равна
	\[ \abs{\vec{B}} = \mu_0 I \frac{N}{l} = \mu_0 I \rho \]
	Где $\rho$ -- плотность намотки, т. е. число витков на единицу длины.
}
\begin{proof}
	TODO
\end{proof}
