\section{Магнитное поле}



\subsection{Основные понятия}
\df{Магнитное поле}{вид материи, который возникает из-за движения заряженных частиц и который обнаруживается по действию на другие движущиеся заряженные частицы, в том числе токи и тела, обладающие магнитным моментом.}
\df{Сила Лоренца}{сила, действующая на движущийся заряд со стороны магнитного поля.}
\df{Вектор магнитной индукции}{силовая характеристика магнитного поля такая, что сила Лоренца со стороны этого поля, действующая на заряд $q$, движущийся со скоростью $\vec{v}$ в данной точке равна}
\[ \vec{F_\text{л}} = q [ \vec{v} \times \vec{B} ] \]
Из определения следует, что единица измерения магнитной индукции -- $\frac{\uN}{\um \cdot \uA}$.\par

Установлено, что сила Лоренца гироскопическая (её работа всегда равна 0) $\thus$ она всегда лежит в плоскости, перпендикулярной вектору скорости частицы $\thus$ такой вектор $\vec{B}$ всегда найдётся.\par

\df{Магнитный момент}{векторная физическая величина, характеризующая магнитные свойства вещества. [$\uA \cdot \um^2$]}

Для плоского контура площадью $S$ с током $I$, вектор нормали для плоскости которого $\vec{n}$ (направление выбирается так, чтобы $\vec{B}_\text{собств.} \upuparrows \vec{n}$), магнитный момент вычисляется как:
\[ \vec{p}_m = I \cdot S \cdot \vec{n} \]

Элементарный контур с током может служить индикатором внешнего магнитного поля. Если размеры контура пренебрежимо малы, то он разворачивается так, чтобы его магнитный момент был сонаправлен с вектором магнитной индуции в данной точке.\par

\df{Сила Ампера}{сила, действующая со стороны магнитного поля на элемент тока.}
\ulaw{
	\[ \dlta \vec{F}_\text{а} = [ I \dlta \vec{l} \times \vec{B}] \]
}
\begin{proof}
	Сила Ампера -- сумма сил Лоренца для всех частиц тока:
	\[ \vec{F_\text{л}} = \dlta q [ \vec{v} \times \vec{B} ] \]
	\[ \vec{F_\text{л}} = [ \dlta q \cdot \vec{v} \times \vec{B} ] \]
	\[ \int \vec{F_\text{л}} = [ \left( \int \dlta q \cdot \vec{v} \right) \times \vec{B} ] \]
	\[ \dlta \vec{F_\text{а}} = [ \left( \int \frac{\dlta q \cdot \dlta \vec{l}}{\dlta t} \right) \times \vec{B} ] \]
	\[ \dlta \vec{F_\text{а}} = [ \left( \int \frac{\dlta q}{\dlta t} \right) \dlta \vec{l} \times \vec{B} ] \]
	\[ \dlta \vec{F_\text{а}} = [ I \dlta \vec{l} \times \vec{B} ] \]
\end{proof}

\law{Закон Био -- Савара (эксперимент)}{
	Вектор магнитной индукции, создаваемый элементом тока $d \vec{l}$, равен
	\[ \dlta \vec{B} = \frac{\mu_0}{4 \pi} \frac{[I \dlta \vec{l} \times \vec{r}]}{r^3} = k_1 \frac{[I \dlta \vec{l} \times \vec{r}]}{r^3} \]
	Где $k_1$ -- коэффициент пропорциональности; $\mu_0$ -- магнитная постоянная, $I$ -- ток, текущий по элементу; $\vec{r}$ -- вектор, началом которого является положение проводника, а концом -- точка, в которой определяется индукция.
	\[ \mu_0 = 1.256\:637\:062\:12(19) \cdot 10^{-6} {\textstyle \frac{\uN}{\um^2}} \text{; } \;
	k_1 = {\displaystyle \frac{\mu_0}{4 \pi}} \approx 10^{-7} {\textstyle \frac{\uN}{\um^2}} \]
}

\ulaw{
	Поток магнитного поля (вектора магнитной индукции) через замкнутую поверхность равен
	\[ \Phi_B = 0 \]
}
\begin{proof}
	TODO?
\end{proof}

\law{Теорема о циркуляции}{
	Циркуляция вектора магнитной индукции по замкнутому контуру равна алгебраической сумме токов, охваченных произвольной поверхностью, натянутой на этот контур, умноженной на магнитную постоянную.
	\[ \oint \vec{B} \dlta \vec{l} = \mu_0 \sum I_i \]
}
\begin{proof}
	TODO?
\end{proof}

\ulaw{
	Индукция магнитного поля бесконечного соленоида на его оси симметрии равна
	\[ \abs{\vec{B}} = \mu_0 I \frac{N}{l} = \mu_0 I \rho \]
	Где $\rho$ -- плотность намотки, т. е. число витков на единицу длины.
}
\begin{proof}
	TODO
\end{proof}



\subsection{Электромагнитная индукция}

Определение потока вектора магнитной индукции через поверхность ничем не отличается от общего определения потока вектора (\ref{fig:flux}).\\
\[ \Phi_B = \oiint_{S} \vec{B} \cdot \dlta \vec{S} = \oiint_{S} \abs{B} \cdot \dlta S \cdot \cos \alpha \]

\df{Поток вектора магнитной индукции для контура}{поток вектора магнитной индукции через любую поверность, натянутую на этот контур. Так как для любой замкнутой поверности $\Phi_B = 0$, то этот поток не зависит от выбранной поверности.}

\begin{wrapfigure}{R}{0.4\textwidth}
	\def\svgwidth{0.4\textwidth}
	\input{images/orientation.pdf_tex}
	\caption{Выбор направления обхода}
	\label{fig:orientation}
\end{wrapfigure}
При определении магнитного потока через поверхность, её требуется ориентировать, после этого ориентируем и контур согласно правилу буравчика (\ref{fig:orientation}).

\df{Собственный магнитный поток контура}{поток, вызываемый током в самом контуре.}
\df{Индуктивность контура}{отношение собственного магнитного потока контура к току, протекающему в нём. [$\uH = \frac{\uV \cdot \us}{\uA}$]}
\[ \Phi_\text{с} = LI \]

\law{Закон электромагнитной индукции}{
	\[ \epsilon = -\frac{\dlta \Phi}{\dlta t} \text{,} \]
	где $\frac{\dlta \Phi}{\dlta t}$ называют скоростью изменения потока.
}
\df{Самоиндукция}{явление возникновения ЭДС в контуре при изменении силы тока или индуктивности.}
\[ \epsilon_\text{с} = -\frac{\dlta \Phi_\text{с}}{\dlta t} = -L \frac{\dlta I}{\dlta t} - \frac{\dlta L}{\dlta t} I \]

\law{Правило Ленца}{
	\begin{enumerate}
		\item Индукционный ток направлен так, чтобы своим магнитным полем противодействовать изменению магнитного потока, которым он вызван.
		\item Направление наведённой ЭДС всегда таково, что она пытается препятствовать причине, её вызывающей.
	\end{enumerate}
}

\begin{wrapfigure}{R}{0.4\textwidth}
	\def\svgwidth{0.4\textwidth}
	\input{images/example-lenz.pdf_tex}
	\caption{Пример применения правила Ленца}
	\label{fig:example-lenz}
\end{wrapfigure}
Рассмотрим пример (\ref{fig:example-lenz}). При удалении магнита, магнитный поток для контура уменьшается, тогда (используя первую формулировку) вектор магнитной индукции тока в контуре будет направлен направо $\thus$ ток течёт против часовой стрелки. (Используя вторую формулировку) Так как причина изменения потока -- удаление магнита, то магнитное поле будет притягивать магнит к кольцу.\par

\mbox{
	\begin{minipage}{0.6\textwidth - \columnsep}
		\law{Закон электромагнитной индукции в формулировке Фарадея}{
			Для контура выполняется
			\[ q = \frac{\Phi_0 - \Phi_1}{R} \text{,} \]
			где $q$ -- ток, прошедший из-за изменения магнитного поля, $R$ -- сопротивление контура.
		}
	\end{minipage}
}

\begin{proof}
	\[ -\frac{\dlta \Phi}{\dlta t} = \epsilon = IR = \frac{\dlta q}{\dlta t}R \]
	\[ -{\dlta \Phi} = {\dlta q}R \]
	\[ \int -{\dlta \Phi} = \int {\dlta q}R \]
	\[ q = \frac{\Phi_0 - \Phi_1}{R} \]
\end{proof}



\subsection{Индукция в движущемся проводнике}

\begin{wrapfigure}{R}{0.2\textwidth}
	\def\svgwidth{0.2\textwidth}
	\input{images/moving-conductor.pdf_tex}
	\caption{Проводник}
	\label{fig:moving-conductor}
\end{wrapfigure}

\mbox{
	\begin{minipage}{0.8\textwidth - \columnsep}
	\ulaw{
		Для проводника, движущегося в магнитном поле справдливо
		\[ \epsilon = \int (\vec{v} \times \vec{B}) \cdot \dlta \vec{l} \]
	}
	\end{minipage}
}

\begin{proof}
	Пусть есть проводник, который движется в магнитном поле. Тогда по определению ЭДС, 
	\[ \epsilon_{AB} = \frac{\int A_\text{ст}}{q} = \frac{\int \vec{F} \cdot \dlta \vec{l}}{q} = \frac{\int q ((\vec{u} + \vec{v}) \times \vec{B}) \cdot \dlta \vec{l}}{q} \text{,} \]
	где $\vec{u}$ -- скорость пробного заряда относительно проводника.
	\[ \epsilon_{AB} = \int \left((\vec{v} \times \vec{B}) \cdot \dlta \vec{l} + (\vec{u} \times \vec{B}) \cdot \dlta \vec{l}\right) \]
	\[ \epsilon_{AB} = \int \left((\vec{v} \times \vec{B}) \cdot \dlta \vec{l} - \vec{B} \cdot (\vec{u} \times \dlta \vec{l})\right) \]
	\[ \epsilon_{AB} = \int \left((\vec{v} \times \vec{B}) \cdot \dlta \vec{l} - \vec{B} \cdot \vec{0} \right) = \int (\vec{v} \times \vec{B}) \cdot \dlta \vec{l} \]
\end{proof}
Если проводник движется поступательно в однородном магнитном поле, то последняя формула превщается в
\[ \epsilon_{AB} = (\vec{v} \times \vec{B}) \cdot \vec{l} \]

Ещё более частные случаи этой формулы. Если проводник прямой и перпендикулярен плоскости, в которой содержатся $\vec{v}$ и $\vec{B}$, а между последними угол $\alpha$, то
\[ \abs{\epsilon_{AB}} = v B l \cdot \sin{\alpha} \]
Если проводник не перпендикулярен, но прямой и находится под углом $\beta$ к плоскости, то
\[ \abs{\epsilon_{AB}} = v B l \cdot \sin{\alpha} \cdot \cos{\beta} = v B l_\text{экв.} \cdot \sin{\alpha} \]



\subsection{Переменное магнитное поле}

Переменное магнитное поле создаёт вихревое электрическое поле.\par
\ulaw{
	Для осесимметричного магнитного поля
	\[ E_\text{вихр.} = \frac{\dlta \Phi}{2 \pi r \dlta t} \text{,} \]
	где $E_\text{вихр.}$ -- напряжённость вихревого электрического поля, $r$ -- расстояние от центра симметрии.
}

\begin{proof}
	Рассмотрим кольцо радиуса $r$ в осесимметричном магнитном поле, причём оси симметрии кольца и поля совпадают. Тогда,
	\[ \epsilon = -\frac{\dlta \Phi}{\dlta t} \]
	\[ \dlta \epsilon_i = \frac{q E_{\text{вихр., }i} \cdot \dlta l_i}{q} = E_{\text{вихр., }i} \cdot \dlta l_i \]
	Т. к. поле симметрично, то все $E_{\text{вихр., }i}$ равны.
	\[ \epsilon = \int E_{\text{вихр., }i} \cdot \dlta l_i = E_\text{вихр.} \cdot 2 \pi r \]
	\[ E_\text{вихр.} = \frac{\dlta \Phi}{2 \pi r \dlta t} \]
\end{proof}
