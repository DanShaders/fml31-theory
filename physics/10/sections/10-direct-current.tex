\section{Постоянный ток}



\subsection{Электронная теория вещества}
\begin{enumerate}
	\item Свободные электроны в метале подчиняются законам идеального газа.
	\item Движение свободных электронов подчиняется законам механики Ньютона.
	\item Электроны не сталкиваются между собой, а сталкиваются только с ионами.
	\item Электроны при столкновении с ионами передают свою кинетическую энергию последним полностью.
\end{enumerate}
Опыты, подтверждающие теорию -- опыт Рикке (1911), Мандельштама и Папалекси (1916), Толмена и Стюарта  (1916).



% \subsection{Основные понятия}
% \df{Электрический ток}{упорядоченное движение заряженных частиц под действием электрического поля.}
% \df{Сила тока}{заряд, протекающий за единицу времени через поперечное сечение проводника. [$\uA$]}
% \[ I = \frac{\dlta q}{\dlta t} \]
% \df{Ампер}{сила тока, которая определена путём фиксации численного значения элементарного заряда равным $1.602\:176\:634 \cdot 10^{-19} \; \uA \cdot \us$.}
% \df{Плотность тока}{векторная физическая величина, сонаправленная с движением положительных зарядов, численно равная отношению силы тока к площади поперечного сечения проводника. [$\frac{\uA}{\um^2}$]}
% \[ \abs{\vec{j}} = \frac{I}{S} \]
% \df{Удельная проводимость}{отношение плотности тока к напряжённости поля в данной точке. [$\frac{\uS}{\um}$]}
% \df{Напряжение}{скалярная физическая величина, равная работе всех сил в электрической цепи по перемещению заряда. В отсутствии сторонних сил напряжение равно разности потенциалов данных двух точек, взятой со знаком минус. [$\uV$]}
% \[ U_{12} = \frac{A_\text{общ.}}{q} = \frac{A_\text{поля}}{q} + \frac{A_\text{внеш}}{q} = (\varphi_1 - \varphi_2) + \eps_12 \]