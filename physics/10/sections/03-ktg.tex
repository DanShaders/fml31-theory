\section{Газы в МКТ}

\df{Основное допущение статистической механики}{среднее по времени значение физическое величины совпадает со статистическим средним в эти моменты времени.}



\subsection{Основное уравнение МКТ}
\law{Основное уравнение МКТ}{
	Для идеального газа выполняется:
	\[p = \frac{1}{3} m_0 \cdot n \cdot \overline{v^2}\]
}
\begin{proof}
	Картинка для визуализации происходящего есть на стр. 112 учебника. Возьмём объём газа и, рассмотрев время $\Delta t$, определим переданный частицами газа стенке импульс:
	\[\displaystyle P = \sum_{v_{ix}>0}{\color{red}P(v_{ix})}=\sum_{v_{ix}>0}(N(v_{ix})\cdot{\color{red}P_0(v_{ix})})=\sum_{v_{ix}>0}({\color{red}N(v_{ix})}\cdot2 m_0 v_{ix})=
	\sum_{v_{ix}>0}(n(v_{ix})\cdot {\color{red}V(v_{ix})}\cdot2 m_0 v_{ix})=
	\]
	\[=\sum_{v_{ix}>0}({\color{red}n(v_{ix})}\cdot S \Delta t {\color{red}v_{ix}}\cdot2 m_0 {\color{red}v_{ix}})=
	(2S \Delta t \cdot m_0) \cdot \sum_{v_{ix}>0}(n(v_{ix})\cdot v_{ix}^2)=
	(2S \Delta t \cdot m_0 \cdot {\color{blue}n}) \cdot \sum_{v_{ix}>0}\frac{n(v_{ix})\cdot v_{ix}^2}{\color{blue}n}=\]
	\[
	=(2S \Delta t \cdot m_0 \cdot n) \cdot {\color{blue}\frac{1}{2}}\sum_{\color{blue}v_{ix}}\frac{n(v_{ix})\cdot v_{ix}^2}{n}=
	S \Delta t \cdot m_0 \cdot n \cdot {\color{blue}\overline{v_{x}^2}}
	\]
	\[F \Delta t = S \Delta t \cdot m_0 \cdot n \cdot \overline{v_{x}^2}\]
	\[F \Delta t = \frac{1}{3} S \Delta t \cdot m_0 \cdot n \cdot \overline{v^2}\]
	\[F = \frac{1}{3} S \cdot m_0 \cdot n \cdot \overline{v^2}\]
	\[p = \frac{1}{3} m_0 \cdot n \cdot \overline{v^2}\]
\end{proof}


\subsection{Следствия из основного уравнения}
По определению средней кинетической энергии молекул ($\displaystyle \overline{E}=\frac{m_0\overline{v^2}}{2}$) несложно показать, что для идеального газа выполняется
\begin{equation}
	p = \frac{2}{3} n \overline{E}
\end{equation}

\df{Постоянная Больцмана}{универсальная газовая постоянная, делённая на число Авогадро.}
\[k=\frac{R}{N_A}=1.380649\cdot10^{-23} \textstyle \frac{\uJ}{\uK}\]

По уравнению Менделеева-Клапейрона можно показать, что
\begin{equation}
	\overline{E}=\frac{3}{2}kT
\end{equation}

Подставив в (3.2.1) (3.2.2), получим
\begin{equation*}
	p=nkT
\end{equation*}

\df{Среднеквадратичная скорость}{квадратный корень из среднего квадрата скорости молекулы.}
\[v_{\text{кв.}}=\sqrt{\overline{v^2}}\]

Заменив кинетическую энергию определением в (2) получим
\begin{equation*}
	v_{\text{кв.}}=\sqrt{\frac{3kT}{m_0}}=\sqrt{\frac{3RT}{M}}
\end{equation*}



\subsection{Распределения скоростей молекул}
\df{Распределение вероятностей}{закон, описывающий область значений случайной величины и соответствующие вероятности появления этих значений.}
\df{Плотность распределения случайной величины $\xi$}{один из способов описания распределения вероятности; функция такая, что если её первообразная $F(x)$, то
$\lim_{x\to+\infty}{F(x)}=1$ и $\forall x_0<x_1 : F(x_1)-F(x_0)=P(x0 \le \xi \le x1)$.}
\df{Наиболее вероятное значение случайной величины}{ордината максимального значения плотности вероятности этой случайной величины.}
\df{Дисперсия случайной величины $\sigma^2$}{среднее значение квадрата случайной величины.}
\df{Нормальное распределение $N(\mu, \sigma^2)$}{распределение, плостность вероятности которого равна
\[f(x)=\frac{1}{\sigma \sqrt{2\pi}}e^{-\frac{(x-\mu)^2}{2\sigma^2}}\]
Для распределения $N(\mu, \sigma^2)$ справедливо, что его среднее и медиана - $\mu$, а дисперсия - $\sigma^2$.}

Т. к. средняя скорость по одной из осей -- это величина, являющаяся средним арифметическим большого количества мало зависимых величин, распределённых примерно одинакого, то по центральной предельной теореме следует, что распределение скоростей стемится к нормальному.\par
$\overline{v_x}=0$, а $\overline{v_x^2}=\frac{p}{m_0 n}=\frac{kT}{m_0}$, значит, что распределение скоростей равно $N(0, \frac{kT}{m_0})$. Значит плотность вероятности равна:
\[f(v)=\sqrt{\frac{m_0}{2\pi kT}}e^{-\frac{m_0 v^2}{2kT}}\]

\df{Распределение Максвелла}{распределение скоростей молекул в идеальном газе.}

Нам придётся поверить, что плотность вероятности распределения Максвелла выглядит как:
\[f(v) = 4\pi(\frac{m_0}{2\pi k T})^{\frac{3}{2}}e^{-\frac{m_0v^2}{2 k T}}v^2\]

Возьмём производную и приравняем её нулю для получения наиболее вероятной скорости:
\[\frac{\text{d}f}{\text{d}v}=8\pi(\frac{m_0}{2\pi k T})^{\frac{3}{2}}ve^{-\frac{m_0v^2}{2 k T}}(-\frac{m_0 v^2}{2kT}+1)\]
\[\frac{\text{d}f}{\text{d}v}=0\]
\[v_{\text{н. в.}}=\sqrt{\frac{2kT}{m_0}}=\sqrt{\frac{2RT}{M}}\]

Утверждение без доказательства для средней скорости:\par
\[\overline{v}=\sqrt{\frac{8kT}{\pi m_0}}=\sqrt{\frac{8RT}{\pi M}}\]



\subsection{Средняя скорость броуновской частицы}
Броуновская частица также участвует в тепловом движении, а значит её средняя кинетическая энергия совпадает с средней энергией частиц, значит (если $m$ -- масса частицы):
\[\overline{v}=\sqrt{\frac{8kT}{\pi m}}\]
\[v_{\text{к. в.}}=\sqrt{\frac{3kT}{m}}\]



\subsection{Внутренняя энергия газа}
\df{Внутренняя энергия газа}{суммарная механическая энергия молекул. [$\uJ$]}
\[U=N \cdot (\overline{E_\text{к}}+\overline{E_\text{п}})\]
Потенциальная энергия неположительна; только у идеального газа равна 0 (т. к. по определению мы считаем, что молекулы не взаимодействуют друг с другом).\par

Для идеального газа справедливо:
\[U=N\cdot E_{\text{к}}=\nu N_A \frac{3}{2}kT=\frac{3}{2} \nu RT\]
\df{Количество степеней свободы}{количество независимых координат, однозначно определяющих положение частицы в пространстве.}

Для газа, у молекул которого $i$ степеней свободы (у одноатомного -- 3, двухатомного (1D молекул) -- 5, остальных (2D молекул) -- 6) средняя кинетическая энергия будет больше в $\frac{i}{3}$ раз, чем для одноатомного газа. Соответственно,
\[ p = \frac{i}{3} n \overline{E} \]
\[ \overline{E}=\frac{i}{2}kT \]
\begin{equation}
U=\frac{i}{2}\nu RT
\end{equation}
Применив закон Менделеева-Клапейрона к (3.5.1), получим:
\begin{equation*}
U=\frac{i}{2}pV
\end{equation*}