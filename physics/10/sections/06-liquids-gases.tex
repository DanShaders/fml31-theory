\section{Жидкости и газы}



\df{Парообразование}{процесс перехода тела из жидкого агрегатного состояния в газообразное.}
\df{Испарение}{процесс парообразования с поверхности жидкости при любой температуре.}
\df{Кипение}{процесс парообразования во всём объеме жидкости, который происходит, если давление насыщенных паров равно давлению жидкости в данной точке.}
\df{Температура кипения}{температура, при которой происходит кипение при данном давлении.}
\df{Динамическое равновесие}{процесс испарения и конденсации, при котором эти процессы компенсируют друг друга.}
\df{Насыщенный пар}{пар, находящийся в динамическом равновесии со своей жидкостью.}
\df{(Абсолютная) максимальная влажность}{плотность насыщенного пара при данной температуре, обычно выраженная в $\frac{\ug}{\umol}.$}
\df{Критическая температура}{минимальная температура, при которой невозможно перевести пар в состояние насыщения.}
\df{Критическая точка}{состояние вещества, при котором нет различий между жидкостью и её насыщенным паром.}
\df{Относительная влажность}{отношение плотности данного пара к плотности насыщенного пара при данной температуре, обычно выраженное в процентах.}
\[ \phi = \frac{p_0}{p_{max}} \]
\df{Точка росы}{температура, до которой должен охладиться пар для достижения насыщения.}
\df{Теплота парообразования}{количество теплоты, необходимое для превращения данной массы жидкости в пар этой же температуры.}
\df{Удельная теплота парообразования}{отношение теплоты парообразования к массе жидкости.}
\[ L = \frac{Q_\text{п}}{m} \]

TODO: факторы кипения и испарения, разные изотермы, [Барометрическая формула].