\section{Электростатика}


\subsection{Основные понятия}
\df{Электрический заряд}{скалярная физическая величина, характеризующая способность тела вступать в электромагнитные взаимодействия.}
\df{Кулон}{заряд, который протекает по проводнику с силой тока $1 \; \uA$ за $1 \; \us$.}
\df{Элементарный электрический заряд}{заряд одного протона или заряд одного электрона, взятый со знаком плюс.}
\[ e = 1.602\:176\:634 \cdot 10^{-19} \; {\textstyle \uC} \]

\law{Закон сохранения заряда}{
	В замкнутой системе $\sum q_i \isconst$.
}

\df{Электризация}{явление перераспределения заряда.}
\df{Ион}{заряженная частица.}
\df{Точечный заряд}{заряженное тело с пренебрежимо малыми размерами.}

\law{Закон Кулона}{
	Для точечных зарядов модуль силы взаимодействия прямо пропорционален модулю зарядов и обратно пропорционален квадрату расстояния между ними.
	\[ F = \frac{1}{4 \pi \eps_0} \frac{|q_1| \cdot |q_2|}{r^2} = k \frac{|q_1| \cdot |q_2|}{r^2}\]
	Где $\eps_0$ -- электрическая постоянная, $k$ -- коэффициент пропорциональности в законе Кулона.
	\[ \eps_0 = 8.854\:187\:812\:8(13) \cdot 10^{-12} \; {\textstyle \frac{\uC^2}{\uN \cdot \um^2}} \text{; }\;
	k = \frac{1}{4 \pi \eps_0} \approx 9 \cdot 10^{9} \; {\textstyle \frac{\uN \cdot \um^2}{\uC^2}} \]
}

\law{Принцип суперпозиции}{
	\begin{enumerate}
		\item При фиксированном распределении зарядов на всех телах силы электростатического (кулоновского) взаимодействия между любыми двумя телами не зависят от наличия других заряженных тел.
		\item Силы кулоновского взаимодействия складываются векторно, не влияя друг на друга.
	\end{enumerate}
}

\df{Напряженность}{векторная физическая величина, силовая характеристика э/м поля, численно равная отношению силы, с которой данное поле в данной точке пространства будет действовать на внесённый пробный точечный положительный заряд к величине этого заряда.}
Напряженность поля с единственным зарядом равна
\[ \vec{E} = \frac{kQ}{r^3} \cdot \vec{r} \]
\df{Силовая линия}{воображаемая линия, к которой напряженность поля в каждой точке, принадлежащей ей, является касательной.}
\begin{enumerate}
	\item Силовые линии не пересекаются.
	\item Силовые линии начинаются и заканчиваются на заряженных телах.
	\item Силовые линии направлены от положительного заряда к отрицательному.
\end{enumerate}
\df{Электрический диполь}{система двух одинаковых по модую разноимённых зарядов, расположенных на расстоянии $l$ друг от друга.}
\df{Дипольный момент}{произведение модуля заряда одной из заряженных частиц диполя на вектор, модуль которого $l$, направленный от отрицательно заряженной частицы к положительно заряженной. [$\uC \cdot \um$]}
\[ \vec{p} = q \vec{l} \]

\df{Поток вектора}{скалярное произведение вектора величины на вектор площади, которую данная величина пронизывает. Т. е. это функция от произольной ориентированной поверхности, возвращающая число.}
\[ \Phi_X = \oiint_{S} \vec{X} \cdot \dlta \vec{S} \]
\textbf{\textit{Замечание. }} $\oiint_{S}$ -- интеграл по всем точкам поверхности $S$, $\oint_{L}$ -- интеграл по всем точкам замкнутой кривой $L$.



\subsection{Теорема Гаусса}
\law{Теорема Гаусса}{
	Поток вектора напряженности через любую замкнутую поверхность пропорционален заключённому внутри этой поверхности электрическому заряду.
	\[ \Phi_E = \frac{1}{\eps_0} \sum q_i \]
}
\begin{proof}
TODO
\end{proof}

TODO: те самые "маленькие теоремы-следствия" из Гаусса

\ulaw{
	Сила нормального действия электрического поля на пластину $S$, которую пронизывает поток $\Phi$, с поверхностной плотностью заряда $\sigma$ равна
	\[ F_n = \Phi \sigma \]
}
\begin{proof}
	\[ \Phi = \oiint_{S} \vec{E} \cdot \dlta \vec{S} = \oiint_{S} \vec{E} \cdot \dlta \vec{S} = \oiint_{S} \abs{\vec{E}} \cdot \dlta S \cdot \cos{\alpha} = \oiint_{S} E_{in} \dlta S = \oiint_{S} \frac{F_{in}}{\sigma \dlta S} \dlta S = \oiint_{S} \frac{F_{in}}{\sigma} = \frac{F_n}{\sigma} \]
	\[ F_n = \Phi \sigma \]
\end{proof}



\subsection{Потенциалы}
\ulaw{
	Работа э/с поля точечного заряда $Q$ при перемещении другого заряда $q$ из положения $A$ в положение $B$ равна
	\[ A = kqQ \left( \frac{1}{r_A} - \frac{1}{r_B} \right) \]
	Где $r_A$ и $r_B$ -- расстояния между зарядами в положениях $A$ и $B$, соответственно.
}
\begin{proof}
	\[ A = \int_A^B F \cdot \dlta \vec{r} = \int_A^B \left( \frac{k \cdot q \cdot Q}{r^3} \vec{r} \right) \cdot \dlta \vec{r} = kqQ \cdot \int_A^B \frac{\vec{r} \dlta \vec{r}}{r^3} = kqQ \left(-\frac{1}{r} \right) \bigg|_{A}^{B} = kqQ \left( \frac{1}{r_A} - \frac{1}{r_B} \right) \]
\end{proof}

Из последней теоремы следует, что работа при перемещении заряда не зависит от траектории $\thus$ кулоновская сила консервативна. Из-за свойства аддитивности э/с поля следует, что этот вывод справедлив для любой системы. Соответственно, можно говорить о потенциальной энергии взаимодействия э/с поля с зарядом.\par

\df{Электростатический потенциал}{скалярная физическая величина, энергетическая характеристика э/с поля, численно равная отношению потенциальной энергии взаимодействия пробного точечного положителельного заряда с этим полем к величине этого заряда. [$\frac{\uJ}{\uC}$, $\uV$]}
\df{Эквипотенциальные поверхности}{поверхности, потенциал в каждой точке которых равен.}

\begin{itemize}
	\item Потенциал в любой точке пространства подчиняется свойству простой аддитивности.
	\item Эквипотенциальные поверхности замкнуты.
	\item Эквипотенциальные поверхности повторяют форму заряда (в частных случаях подобны).
\end{itemize}

По определению, потенциал равен работе поля по перемещению заряда из данной точки в точку с нулевым потенциалом, делённую на величину этого заряда. Обычно за нулевой уровень потенциальной энергии принимается точка, бесконечено удалённая от системы. 

\ulaw{
	Потенциал точки на расстоянии $r$ от уединённого точечного заряда $Q$ равен
	\[ \varphi = \frac{kQ}{r} \]
}
\begin{proof}
	\[ \varphi = \frac{E_\text{п}}{q} = \frac{-A_{+\infty \to r}}{q} = -\frac{kQq}{q}\left( \frac{1}{+\infty} - \frac{1}{r} \right) = \frac{kQ}{r} \]
\end{proof}

\df{Градиент}{вектор, своим направлением указывающий направление наибольшего возрастания некоторой скалярной величины, модуль которого пропорционален скорости роста этой величины.}
\[ (\grad f)(x, y, z) = \left( \frac{\partial f}{\partial x},\;\frac{\partial f}{\partial y},\;\frac{\partial f}{\partial z} \right) \]

\ulaw{
	Напряженность э/с поля равна минус градиенту потенциала этого поля.
	\[ E = -\grad \varphi \]
}
\begin{proof}
	Понятно, что из-за аддитивности градиента (из-за аддитивности частной производной) и э/с поля, для доказательства теоремы достаточно показать справедливость равенства для поля уединённого точечного заряда.\par
	Введём систему координат так, чтобы её центр совпадал с точечным зарядом, тогда:
	\[ \frac{\partial \varphi}{\partial x} = \frac{\dlta}{\dlta x} \left( \frac{kQ}{r} \right) = \frac{\dlta}{\dlta x} \left( \frac{kQ}{\sqrt{x^2+y^2+z^2}} \right) = \frac{kQ}{-2 \left( x^2 + y^2 + z^2 \right)^{1.5}} \cdot 2x = \frac{kQ}{-\left( r^2 \right)^{1.5}} \cdot x = -\frac{kQ}{r^3} \cdot x\]
	Аналогично для производных по $y$ и $z$.
	\[ -\grad \varphi = -\left( \frac{\partial f}{\partial x},\;\frac{\partial f}{\partial y},\;\frac{\partial f}{\partial z} \right) = -\left( -\frac{kQ}{r^3} \cdot x,\; -\frac{kQ}{r^3} \cdot y,\; -\frac{kQ}{r^3} \cdot z \right) = \frac{kQ}{r^3} \cdot (x,\; y,\; z) = \frac{kQ}{r^3} \cdot \vec{r} = E \]
\end{proof}

Т. к. линии уровня функции перпендикулярны градиенту, а силовые линии параллельны градиенту, то эквипотенциальные поверхности в любой точке перпендикулярны соответствующим силовым линиям э/с поля.

\ulaw{
	Если принять нулевой уровень потенциальной энергии на расстоянии $q$ от бесконечной равномерно заряженной пластины, то потенциал в точке на расстоянии $x$ равен
	\[ \varphi(x) = -\frac{\sigma (x - q)}{2 \eps_0} \]
	В частности, если нулевой уровень находится на плоскости пластины, то $ \varphi(x) = -\frac{\sigma x}{2 \eps_0} $.
}
\begin{proof}
	TODO
\end{proof}

\ulaw{
	Если принять нулевой уровень потенциальной энергии на расстоянии $q$ от бесконечной равномерно заряженной нити, то потенциал в точке на расстоянии $x$ равен
	\[ \varphi(x) = \frac{k \tau}{2} \cdot \ln{\frac{q}{x}} \]
}
\begin{proof}
	TODO
\end{proof}

\df{Напряжение между двумя точками}{работа всех сил по перемещению пробного заряда из одной точки в другую, делённая на величину этого заряда. [$\uV$]}



\subsection{Диэлектрики в электрическом поле}
Из-за особенностей своей структуры (кажется, это называется "поляризация"), диэлектрики изменяют внешнее электрическое поле. В некотором приближении, они только ослабляют поле внутри себя. Отметим, что это не всегда работает так (даже в нашей модели), потому что наличие таких областей с ослабленным полем нарушает утверждение о консервативности э/с силы (если диэлектрик не центрально симметричен).\par
\df{Относительная диэлектрическая проницаемость}{скалярная величина, показывающая во сколько раз диэлектрик ослабляет внешнее электрическое поле внутри себя.}
\[ \eps = \frac{\abs{E_\text{внешнее}}}{\abs{E_\text{результ.}}} \]

Если диэлектрик занимает всё пространство, то это значит, что результирующие силы, действующая на заряженные частицы, будут в $\eps$ раз меньше, а значит, потенциал и напряженность в каждой точке тоже будут в $\eps$ раз меньше.\par



\subsection{Электрическая ёмкость}
В каждой точке проводника потенциал одинаков (иначе бы там был ток) $\thus$ потенциалом проводника называют потенциал в любой его точке.\par

\df{Электрическая ёмкость}{отношение заряда проводника к его потенциалу. [$\frac{\uC}{\uV}$, $\uF$]}
\df{Фарад}{электрическая ёмкость, такая что проводник ёмкостью $1 \; \uF$ имеет потенциал $1 \; \uV$ при заряде $1 \; \uC$.}
\df{Ёмкость конденсатора}{отношение заряда на одной из обкладок конденсатора к напряжению между этой обкладкой и другой.}
\[ C = \frac{q}{\varphi}; \quad C = \frac{q}{U} \]

Т. к. $\varphi$ и $U$ уменьшаются в $\eps$ раз, если всё пространство заполнено диэлектриком, то $C$ увеличивается в $\eps$ раз в этом случае.\par
Величина заряда на обкладках конденсатора, находящего в цепи, всегда равна по модулю и противоположна по знаку (у меня до сих пор нет нормального объяснения).

\ulaw{
	Ёмкость проводящего шара радиусом $r$ равна
	\[ C = \frac{\eps r}{k} \]
}
\begin{proof}
	\[ C = \frac{q}{\varphi} = \frac{\eps qr}{kq} = \frac{\eps r}{k} \]
\end{proof}

\ulaw{
	Ёмкость плоского конденсатора, у которого площадь обкладок $S$, расстояние между ними $d$, равна
	\[ C = \frac{\eps \eps_0 S}{d} \]
}
\begin{proof}
	Примем ноль потенциальной энергии на одной из обладок, тогда
	\[ \varphi_2 = -\frac{\sigma d}{2 \eps \eps_0} + \frac{-\sigma d}{2 \eps \eps_0} = -\frac{\sigma d}{\eps \eps_0}\]
	\[ C = \frac{q}{U} = \frac{q}{\varphi_1 - \varphi_2} = \frac{q}{0 - \varphi_2} = \frac{\eps \eps_0 q}{\sigma d} = \frac{\eps \eps_0 q}{\frac{q}{S} d} = \frac{\eps \eps_0 S}{d} \]
\end{proof}

\ulaw{
	Ёмкость сферического конденсатора с радиусами обкладок $r_1 > r_2$ равна
	\[ C = \frac{\eps r_1 r_2}{k (r_1 - r_2)} \]
}
\begin{proof}
	\[ \varphi_1 = \frac{k q}{\eps r_1} + \frac{k (-q)}{\eps r_1} = 0 \]
	\[ \varphi_2 = \frac{k q}{\eps r_1} + \frac{k (-q)}{\eps r_2} \]
	\[ C = \frac{q}{U} = \frac{q}{\varphi_1 - \varphi_2} = \frac{q}{-\frac{k q}{\eps r_1} + \frac{k q}{\eps r_2}} = \frac{\eps}{k \left( \frac{1}{r_2} - \frac{1}{r_1} \right)} = \frac{\eps r_1 r_2}{k (r_1 - r_2)} \]
\end{proof}

\df{Энергия конденсатора}{работа электрического поля при перемещении избыточного заряда с одной обкладки на другую. [$\uJ$]}

\ulaw{
	Для конденсатора с зарядом на обладке $q$, напряжением между обкладками $U$ и ёмкостью $C$ выполяется
	\[ W_\text{п} = \frac{qU}{2} = \frac{CU^2}{2} = \frac{q^2}{2C} \]
}
\begin{proof}
	Пусть $a$ - потеницал, создаваемый отрицально заряженной обкладкой в точках положительно заряженной обкладки, если принять нулевой уровень на отрицальной обкладке. Т. к. обкладки обладают одинаковым по модулю зарядом, то они вносят равный вклад в разность потенциалов между ними $\thus a = \frac{U}{2}$.

	\[ W_\text{п} = A_\text{поля} = -\Delta E_\text{п} = -(0 - q \cdot a) = q a = \frac{qU}{2} = \frac{q \frac{q}{C}}{2} = \frac{q^2}{2C} \]
	\[ W_\text{п} = \frac{qU}{2} = \frac{(CU)U}{2} = \frac{CU^2}{2} \]
\end{proof}

\df{Объёмная плотность энергии}{отношение энергии конденсатора к объёму, занимаемому им. [$\frac{\uJ}{\um^3}$]}
\[ \omega_e = \frac{W_\text{п}}{V} \]

\ulaw{
	Объёмная плотность энергии в конденсаторе
	\[ \omega_e = \frac{\eps \eps_0 E^2}{2} \]
}
\begin{proof}
	TODO
\end{proof}