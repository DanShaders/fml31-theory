\section{Электростатика}


\subsection{Основные понятия}
\df{Электрический заряд}{скалярная физическая величина, характеризующая способность тела вступать в электромагнитные взаимодействия.}
\df{Кулон}{заряд, который протекает по проводнику с силой тока $1 \; \uA$ за $1 \; \us$.}
\df{Элементарный электрический заряд}{заряд одного протона или заряд одного электрона, взятый со знаком плюс.}
\[ e = 1.602\:176\:634 \cdot 10^{-19} \; {\textstyle \uC} \]

\law{Закон сохранения заряда}{
	В замкнутой системе $\sum q_i \isconst$.
}

\df{Электризация}{явление перераспределения заряда.}
\df{Ион}{заряженная частица.}
\df{Точечный заряд}{заряженное тело с пренебрежимо малыми размерами.}

\law{Закон Кулона}{
	Для точечных зарядов модуль силы взаимодействия прямо пропорционален модулю зарядов и обратно пропорционален квадрату расстояния между ними.
	\[ F = \frac{1}{4 \pi \varepsilon_0} \frac{|q_1| \cdot |q_2|}{r^2} = k \frac{|q_1| \cdot |q_2|}{r^2}\]
	Где $\varepsilon_0$ -- электрическая постоянная, $k$ -- коэффициент пропорциональности в законе Кулона.
	\[ \varepsilon_0 = 8.854\:187\:812\:8(13) \cdot 10^{-12} \; {\textstyle \frac{\uC^2}{\uN \cdot \um^2}} \text{; }\;
	k = \frac{1}{4 \pi \varepsilon_0} \approx 9 \cdot 10^{9} \; {\textstyle \frac{\uN \cdot \um^2}{\uC^2}} \]
}

\law{Принцип суперпозиции}{
	\begin{enumerate}
		\item При фиксированном распределении зарядов на всех телах силы электростатического (кулоновского) взаимодействия между любыми двумя телами не зависят от наличия других заряженных тел.
		\item Силы кулоновского взаимодействия складываются векторно, не влияя друг на друга.
	\end{enumerate}
}

\df{Напряженность}{векторная физическая величина, силовая характеристика э/м поля, численно равная отношению силы, с которой данное поле в данной точке пространства будет действовать на внесённый пробный точечный положительный заряд к величине этого заряда.}
Напряженность поля с единственным зарядом равна
\[ \vec{E} = \frac{kQ}{r^3} \cdot \vec{r} \]
\df{Силовая линия}{воображаемая линия, к которой напряженность поля в каждой точке, принадлежащей ей, является касательной.}
\begin{enumerate}
	\item Силовые линии не пересекаются.
	\item Силовые линии начинаются и заканчиваются на заряженных телах.
	\item Силовые линии направлены от положительного заряда к отрицательному.
\end{enumerate}
\df{Электрический диполь}{система двух одинаковых по модую разноимённых зарядов, расположенных на расстоянии $l$ друг от друга.}
\df{Дипольный момент}{произведение модуля заряда одной из заряженных частиц диполя на вектор, модуль которого $l$, направленный от отрицательно заряженной частицы к положительно заряженной. [$\uC \cdot \um$]}
\[ \vec{p} = q \vec{l} \]

\df{Поток вектора}{скалярное произведение вектора величины на вектор площади, которую данная величина пронизывает. Т. е. это функция от произольной ориентированной поверхности, возвращающая число.}
\[ \Phi_X = \oiint_{S} \vec{X} \cdot \dlta \vec{S} \]
\textbf{\textit{Замечание. }} $\oiint_{S}$ -- интеграл по всем точкам поверхности $S$, $\oint_{L}$ -- интеграл по всем точкам замкнутой кривой $L$.



\subsection{Теорема Гаусса}
\law{Теорема Гаусса}{
	Поток вектора напряженности через любую замкнутую поверхность пропорционален заключённому внутри этой поверхности электрическому заряду.
	\[ \Phi_E = \frac{1}{\varepsilon_0} \sum q_i \]
}
\begin{proof}
TODO
\end{proof}

TODO: те самые "маленькие теоремы-следствия" из Гаусса

\ulaw{
	Сила нормального действия электрического поля на пластину $S$, которую пронизывает поток $\Phi$, с поверхностной плотностью заряда $\sigma$ равна
	\[ F_n = \Phi \sigma \]
}
\begin{proof}
	\[ \Phi = \oiint_{S} \vec{E} \cdot \dlta \vec{S} = \oiint_{S} \vec{E} \cdot \dlta \vec{S} = \oiint_{S} \abs{\vec{E}} \cdot \dlta S \cdot \cos{\alpha} = \oiint_{S} E_{in} \dlta S = \oiint_{S} \frac{F_{in}}{\sigma \dlta S} \dlta S = \oiint_{S} \frac{F_{in}}{\sigma} = \frac{F_n}{\sigma} \]
	\[ F_n = \Phi \sigma \]
\end{proof}



\subsection{Потенциалы}
\ulaw{
	Работа э/с поля точечного заряда $Q$ при перемещении другого заряда $q$ из положения $A$ в положение $B$ равна
	\[ A = kqQ \left( \frac{1}{r_A} - \frac{1}{r_B} \right) \]
	Где $r_A$ и $r_B$ -- расстояния между зарядами в положениях $A$ и $B$, соответственно.
}
\begin{proof}
	\[ A = \int_A^B F \cdot \dlta \vec{r} = \int_A^B \left( \frac{k \cdot q \cdot Q}{r^3} \vec{r} \right) \cdot \dlta \vec{r} = kqQ \cdot \int_A^B \frac{\vec{r} \dlta \vec{r}}{r^3} = kqQ \left(-\frac{1}{r} \right) \bigg|_{A}^{B} = kqQ \left( \frac{1}{r_A} - \frac{1}{r_B} \right) \]
\end{proof}

Из последней теоремы следует, что работа при перемещении заряда не зависит от траектории $\thus$ кулоновская сила консервативна. Из-за свойства аддитивности э/с поля следует, что этот вывод справедлив для любой системы. Соответственно, можно говорить о потенциальной энергии взаимодействия э/с поля с зарядом.\par

\df{Электростатический потенциал}{скалярная физическая величина, энергетическая характеристика э/с поля, численно равная отношению потенциальной энергии взаимодействия пробного точечного положителельного заряда с этим полем к величине этого заряда. [$\frac{\uJ}{\uC}$, $\uV$]}
\df{Эквипотенциальные поверхности}{поверхности, потенциал в каждой точке которых равен.}

\begin{itemize}
	\item Потенциал в любой точке пространства подчиняется свойству простой адитивности.
	\item Эквипотенциальные поверхности замкнуты.
	\item Эквипотенциальные поверхности повторяют форму заряда (в частных случаях подобны).
\end{itemize}

Если за нулевой уровень потенциальной энергии принята точка, бесконечено удалённая от системы (как обычно), тогда потенциал равен работе поля по перемещению заряда из данной точки в бесконечно удалённую, делённую на величину этого заряда.

\ulaw{
	Потенциал точки на расстоянии $r$ от уединённого точечного заряда $Q$ равен
	\[ \varphi = \frac{kQ}{r} \]
}
\begin{proof}
	\[ \varphi = \frac{E_\text{п}}{q} = \frac{-A_{+\infty \to r}}{q} = -\frac{kQq}{q}\left( \frac{1}{+\infty} - \frac{1}{r} \right) = \frac{kQ}{r} \]
\end{proof}

\df{Градиент}{вектор, своим направлением указывающий направление наибольшего возрастания некоторой скалярной величины, модуль которого пропорционален скорости роста этой величины.}
\[ (\grad f)(x, y, z) = \left( \frac{\partial f}{\partial x},\;\frac{\partial f}{\partial y},\;\frac{\partial f}{\partial z} \right) \]

\ulaw{
	Напряженность э/с поля равна минус градиенту потенциала этого поля.
	\[ E = -\grad \varphi \]
}
\begin{proof}
	Понятно, что из-за аддитивности градиента (из-за аддитивности частной производной) и э/с поля, для доказательства теоремы достаточно показать справедливость равенства для поля уединённого точечного заряда.\par
	Введём систему координат так, чтобы её центр совпадал с точечным зарядом, тогда:
	\[ \frac{\partial \varphi}{\partial x} = \frac{\dlta}{\dlta x} \left( \frac{kQ}{r} \right) = \frac{\dlta}{\dlta x} \left( \frac{kQ}{\sqrt{x^2+y^2+z^2}} \right) = \frac{kQ}{-2 \left( x^2 + y^2 + z^2 \right)^{1.5}} \cdot 2x = \frac{kQ}{-\left( r^2 \right)^{1.5}} \cdot x = -\frac{kQ}{r^3} \cdot x\]
	Аналогично для производных по $y$ и $z$.
	\[ -\grad \varphi = -\left( \frac{\partial f}{\partial x},\;\frac{\partial f}{\partial y},\;\frac{\partial f}{\partial z} \right) = -\left( -\frac{kQ}{r^3} \cdot x,\; -\frac{kQ}{r^3} \cdot y,\; -\frac{kQ}{r^3} \cdot z \right) = \frac{kQ}{r^3} \cdot (x,\; y,\; z) = \frac{kQ}{r^3} \cdot \vec{r} = E \]
\end{proof}

Т. к. линии уровня функции перпендикулярны градиенту, а силовые линии параллельны градиенту, то эквипотенциальные поверхности в любой точке перпендикулярны соответствующим силовым линия э/с поля.

TODO:
Напряжение\par
Работа в однородном поле: $U = qE(x_1 - x_2)$ \par
Работа в неоднородном, центральносимметричном поле \par
