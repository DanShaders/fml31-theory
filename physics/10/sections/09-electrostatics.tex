\section{Электростатика}


\subsection{Основные понятия}
\df{Электрический заряд}{скалярная физическая величина, характеризующая способность тела вступать в электромагнитные взаимодействия.}
\df{Кулон}{заряд, который протекает по проводнику с силой тока $1 \; \uA$ за $1 \; \us$.}
\df{Элементарный электрический заряд}{заряд одного протона или заряд одного электрона, взятый со знаком плюс.}
\[ e = 1.602\:176\:634 \cdot 10^{-19} \; {\textstyle \uC} \]

\law{Закон сохранения заряда}{
	В замкнутой системе $\sum q_i \isconst$.
}

\df{Электризация}{явление перераспределения заряда.}
\df{Ион}{заряженная частица.}
\df{Точечный заряд}{заряженное тело с пренебрежимо малыми размерами.}

\law{Закон Кулона}{
	Для точечных зарядов модуль силы взаимодействия прямо пропорционален модулю зарядов и обратно пропорционален квадрату расстояния между ними.
	\[ F = \frac{1}{4 \pi \varepsilon_0} \frac{|q_1| \cdot |q_2|}{r^2} = k \frac{|q_1| \cdot |q_2|}{r^2}\]
	Где $\varepsilon_0$ -- электрическая постоянная, $k$ -- коэффициент пропорциональности в законе Кулона.
	\[ \varepsilon_0 = 8.854\:187\:812\:8(13) \cdot 10^{-12} \; {\textstyle \frac{\uC^2}{\uN \cdot \um^2}} \text{; }\;
	k = \frac{1}{4 \pi \varepsilon_0} \approx 9 \cdot 10^{9} \; {\textstyle \frac{\uN \cdot \um^2}{\uC^2}} \]
}

\law{Принцип суперпозиции}{
	\begin{enumerate}
		\item При фиксированном распределении зарядов на всех телах силы электростатического (кулоновского) взаимодействия между любыми двумя телами не зависят от наличия других заряженных тел.
		\item Силы кулоновского взаимодействия складываются векторно, не влияя друг на друга.
	\end{enumerate}
}

\df{Напряженность}{векторная физическая величина, силовая характеристика э/м поля, численно равная отношению силы, с которой данное поле в данной точке пространства будет действовать на внесённый пробный точечный положительный заряд к величине этого заряда.}
Напряженность поля с единственным зарядом равна
\[ \vec{E} = \frac{kQ}{r^3} \cdot \vec{r} \]
\df{Силовая линия}{воображаемая линия, к которой напряженность поля в каждой точке, принадлежащей ей, является касательной.}
\begin{enumerate}
	\item Силовые линии не пересекаются.
	\item Силовые линии начинаются и заканчиваются на заряженных телах.
	\item Силовые линии направлены от положительного заряда к отрицательному.
\end{enumerate}
\df{Электрический диполь}{система двух одинаковых по модую разноимённых зарядов, расположенных на расстоянии $l$ друг от друга.}
\df{Дипольный момент}{произведение модуля заряда одной из заряженных частиц диполя на вектор, модуль которого $l$, направленный от отрицательно заряженной частицы к положительно заряженной. [$\uC \cdot \um$]}
\[ \vec{p} = q \vec{l} \]
\df{Поток вектора}{скалярное произведение вектора величины на вектор площади, которую данная величина пронизывает. Например, для потока вектора напряженности:}
\[ \Phi_E = \oint_{S} \vec{E} \cdot d\vec{S} \]


\subsection{Теорема Гаусса}
\law{Теорема Гаусса}{
	Поток вектора напряженности через любую замкнутую поверхность пропорционален заключённому внутри этой поверхности электрическому заряду.
	\[ \Phi_E = \frac{1}{\varepsilon_0} \sum q_i \]
}
\begin{proof}
TODO
\end{proof}