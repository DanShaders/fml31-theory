\section{Температура. Газовые законы}



\subsection{Основые определения}
\textbf{Термодинамическая система} - любое макроскопическое тело или группа макроскопических тел.\par
\textbf{Макроскопические параметры} - величины, характеризующие состояние термодинамической системы без учета молекулярного строения тел.\par
\textbf{Идеальный газ} - газ, который в точности подчиняется газовым законам.\par
\textbf{Тепловое равновесие (равновесное термодинамическое состояние)} - состояние, при котором все макроскопические параметры тела сколь угодно долго остаются неизменными.\par
\textbf{Температура} - мера средней кинетической энергии молекул. [$^\circ$C, $^\circ$K, $^\circ$F] Если в системе все тела имеют одинаковую температуру, то она будет находиться в состоянии теплового равновесия.\par
\textbf{Аболютный ноль} - температура, при которой прекращается тепловое движение.\par

\textbf{Термодинамический процесс} - процесс, при котором изменяются термодинамические парамеры.\par
\textbf{Равновесный процесс} - процесс, при котором система проходит непрерывный ряд бесконечно близких равновесных термодинамических состояний.\par
\textbf{Неравновесный процесс} - процесс, при котором в системе нарушается тепловое равновесие.\par
\textbf{Время релаксации} - время, в течение которого система после совершения неравновесного процесса возвращается в состояние теплого равновесия.\par



\subsection{Процессы изменения состояния термодинамической системы}
\textbf{Изотермический процесс} - процесс изменения состояния термодинамической системы при постоянной температуре.\par
\textbf{Изобарный процесс} - процесс изменения состояния термодинамической системы при постоянном давлении.\par
\textbf{Изохорный процесс} - процесс изменения состояния термодинамической системы при постоянном объёме.\par
\textbf{Парциальное давление} - давление, которое бы имел бы газ из смеси, если удалить из объёма остальные газы.\par
\begin{oframed}
\textbf{Закон Бойля-Мариотта (эксперимент)}\par
При постоянной температуре и массе произведение давления газа на его объём постоянно.
\[pV=\text{const}\]
\end{oframed}
\begin{oframed}
\textbf{Закон Гей-Люссака (эксперимент)}\par
При постоянном давлении и массе относительное изменение объёма газа прямо пропорциально изменению температуры.
\[\frac{\Delta V}{V}=\alpha \Delta t\]
\[\frac{V_1}{V_2}=\frac{T_1}{T_2}\]
\end{oframed}
\begin{oframed}
\textbf{Закон Авогадро (эксперимент)}\par
Различные газы, взятые в количестве 1 моль, имеют одинаковые объёмы при одинаковых давлениях и температуре.
\end{oframed}
\begin{oframed}
\textbf{Закон Дальтона (эксперимент)}\par
Для достаточно разреженных газов давление смеси газов равно сумме парциальных давлений газов смеси.
\[p=\sum p_i\]
\end{oframed}



\subsection{Состояние идеального газа}
\begin{oframed}
\textbf{Уравнение Клапейрона}\par
Для данной массы газа произведение давления газа данной массы на его объём, деленное на термодинамическую температуру, постоянно.
\[\frac{pV}{T}=\text{const}\]
\end{oframed}
Разобьём процесс, который происходит над газом на два: первый - изобарное изменение объёма и температуры и второй - изотермическое изменение объёма и давления.\par
$\displaystyle \begin{cases}
	p_1 = p_2 \\
	T_2 = T_3 \\
	\frac{V_1}{V_2}=\frac{T_1}{T_2} \text{ (по закону Гей-Люссака для процесса 1)} \\
	p_2 V_2 = p_3 V_3 \text{ (по закону Бойля-Мариотта для процесса 2)}
\end{cases}$\par\medskip
$\displaystyle \begin{cases}
	\frac{V_1}{V_2}=\frac{T_1}{T_3} \\
	p_1 V_2 = p_3 V_3
\end{cases}$\par\medskip
$\displaystyle \begin{cases}
	V_2=\frac{V_1 T_3}{T_1} \\
	V_2=\frac{p_3 V_3}{p_1}
\end{cases}$\par\medskip
$\displaystyle \frac{p_3 V_3}{p_1}=\frac{V_1 T_3}{T_1}$\par\medskip
$\displaystyle \frac{p_3 V_3}{T_3}=\frac{p_1 V_1}{T_1}$ - ЧТД\par
\bigskip
\textbf{Универсальная газовая постоянная} - произведение давления газа на его молярный объём, делённое на температуру. (Скорее всего, надо уметь показывать, что для разных газов это значение одинакого через закон Авогадро)\par
$R = 8.31145 \textstyle \frac{\text{Дж}}{\text{моль} \cdot ^\circ \text{К}}$

\begin{oframed}
\textbf{Уравнение Менделеева-Клапейрона}\\
Для газа выполняется:
\[pV=\nu R T\]
\end{oframed}
По определению $R$:
\[R=\frac{pV_M}{T}\]
\[R=\frac{pV}{\nu T}\]
\[pV = \nu R T\]

\begin{oframed}
\textbf{Закон Шарля}\\
При постоянном объёме и массе давление газа прямо пропорциально абсолютной температуре.
\end{oframed}
По уравнению Менделеева-Клапейрона:
\[pV=\nu R T\]
\[p=\frac{\nu R}{V} T\]
\[p=\frac{m R}{M V} T\]