\section{Температура. Газовые законы}



\subsection{Основые определения}
\df{Термодинамическая система}{макроскопическое тело или группа макроскопических тел.}
\df{Макроскопические параметры}{величины, характеризующие состояние термодинамической системы без учета молекулярного строения тел.}
\df{Идеальный газ}{газ, который в точности подчиняется газовым законам.}
\df{Тепловое равновесие (равновесное термодинамическое состояние)}{состояние, при котором все макроскопические параметры тела сколь угодно долго остаются неизменными.}
\df{Температура}{мера средней кинетической энергии молекул. [$^\circ$C, $\uK$, $^\circ$F] Если в системе все тела имеют одинаковую температуру, то она будет находиться в состоянии теплового равновесия.}
\df{Аболютный ноль}{температура, при которой прекращается тепловое движение.}

\df{Термодинамический процесс}{процесс, при котором изменяются термодинамические парамеры.}
\df{Равновесный (квазистатический, квазиравновесный) процесс}{процесс, при котором система проходит непрерывный ряд бесконечно близких равновесных термодинамических состояний.}
\df{Неравновесный процесс}{процесс, при котором в системе нарушается тепловое равновесие.}
\df{Время релаксации}{время, в течение которого система после совершения неравновесного процесса возвращается в состояние теплого равновесия.}
\df{Квазистатический процесс}{относительно медленный процесс, длительность протекания которого много больше времени релаксации системы.}
Только квазистатические процессы можно изображать на $pV$ диаграмме сплошной линией.\par



\subsection{Процессы изменения состояния термодинамической системы}
\df{Изотермический процесс}{процесс изменения состояния термодинамической системы при постоянной температуре.}
\df{Изобарный процесс}{процесс изменения состояния термодинамической системы при постоянном давлении.}
\df{Изохорный процесс}{процесс изменения состояния термодинамической системы при постоянном объёме.}
\df{Парциальное давление}{давление, которое бы имел бы газ из смеси, если удалить из объёма остальные газы.}
\law{Закон Бойля-Мариотта (эксперимент)}{
	При постоянной температуре и массе произведение давления газа на его объём постоянно.
	\[pV=\isconst\]
}
\law{Закон Гей-Люссака (эксперимент)}{
	При постоянном давлении и массе относительное изменение объёма газа прямо пропорциально изменению температуры.
	\[\frac{\Delta V}{V}=\alpha \Delta t \text{ или } \frac{V_1}{V_2}=\frac{T_1}{T_2}\]
}
\law{Закон Авогадро (эксперимент)}{
	Различные газы, взятые в количестве 1 моль, имеют одинаковые объёмы при одинаковых давлениях и температуре.
}
\law{Закон Дальтона (эксперимент)}{
	Для достаточно разреженных газов давление смеси газов равно сумме парциальных давлений газов смеси.
	\[p=\sum p_i\]
}



\subsection{Состояние идеального газа}
\law{Уравнение Клапейрона}{
	Для данной массы газа произведение давления газа данной массы на его объём, деленное на термодинамическую температуру, постоянно.
	\[\frac{pV}{T}=\isconst\]
}
\begin{proof}
	Разобьём процесс, который происходит над газом на два: первый - изобарное изменение объёма и температуры и второй - изотермическое изменение объёма и давления.\par
	$\displaystyle \begin{cases}
		p_1 = p_2 \\
		T_2 = T_3 \\
		\frac{V_1}{V_2}=\frac{T_1}{T_2} \text{ (по закону Гей-Люссака для процесса 1)} \\
		p_2 V_2 = p_3 V_3 \text{ (по закону Бойля-Мариотта для процесса 2)}
	\end{cases}$\par\medskip
	$\displaystyle \begin{cases}
		\frac{V_1}{V_2}=\frac{T_1}{T_3} \\
		p_1 V_2 = p_3 V_3
	\end{cases}$\par\medskip
	$\displaystyle \begin{cases}
		V_2=\frac{V_1 T_3}{T_1} \\
		V_2=\frac{p_3 V_3}{p_1}
	\end{cases}$\par\medskip
	$\displaystyle \frac{p_3 V_3}{p_1}=\frac{V_1 T_3}{T_1}$\par\medskip
	$\displaystyle \frac{p_3 V_3}{T_3}=\frac{p_1 V_1}{T_1}$\par
\end{proof}

\df{Универсальная газовая постоянная}{произведение давления газа на его молярный объём, делённое на температуру. (Скорее всего, надо уметь показывать, что для разных газов это значение одинакого через закон Авогадро)}
\[R = 8.31145 \textstyle \frac{\uJ}{\umol \cdot \uK}\]

\law{Уравнение Менделеева-Клапейрона}{
	Для газа выполняется:
	\[pV=\nu R T\]
}
\begin{proof}
	По определению $R$:
	\[R=\frac{pV_M}{T}\]
	\[R=\frac{pV}{\nu T}\]
	\[pV = \nu R T\]
\end{proof}

\law{Закон Шарля}{
	При постоянном объёме и массе давление газа прямо пропорциально абсолютной температуре.
}
\begin{proof}
	По уравнению Менделеева-Клапейрона:
	\[pV=\nu R T\]
	\[p=\frac{\nu R}{V} T\]
	\[p=\frac{m R}{M V} T\]
\end{proof}