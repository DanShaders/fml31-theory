\section{Термодинамика}



\subsection{Работа газа}
\ulaw{
	Для любого газа выполняется:
	\begin{equation}
	A=\int_{A}^{B}{p\cdot \mathrm{d}V}
	\end{equation}
}

Геометрический смысл (4.1.1): работа газа равна площади под графиком в координатах $p$ от $V$.\par
Очевидно, что при сжатии газа работа газа отрицательна; при расширении - положительна.\par



\subsection{Первое начало термодинамики}
\df{Теплота}{энергия, которой могут обмениваться тела без совершения работы.}
\df{Адиабатический процесс}{процесс, проходящий без обмена теплом с окружающей средой.}
Быстропротекающие процессы и процессы, проходящие в теплоизолированных сосудах, как правило, адиабатические.\par
\law{Первое начало термодинамики}{
	Количество тепла, которое передаётся телу, расходуется им на изменение внутренней энергии и совершение работы.
	\begin{equation*}
	\delta Q = \dlta U + \delta A
	\end{equation*}
}



\subsection{Второе начало термодинамики}
\df{Переменная состояния}{физическая величина, характеризующая состояние термодинамической системы, и допускаящая измерение с заданной точностью.}
\df{Функция состояния}{физическая величина, рассматриваемая как функция нескольких независимых переменных состояния, которая зависит только от текущего состояния системы, а не пути термодинамического процесса.}
\df{Обратимый процесс}{процесс перехода из начального состояния в конечное, если возможно вернуть систему хотя бы одним способом в исходное состояние, причём так, чтобы во всех остальных телах не произошло изменений.}
Процесс распределения части по всему объёму сосуда из его части необратим.\par
\df{Энтропия}{мера необратимости рассеяния энергии; функция состояния системы, равная отношению приобретённой (или потерянной) энергии к термодинамической температуре. [$\frac{\uJ}{\uK}$]}
\[\dlta S = \frac{\delta Q}{T} \]
В состоянии термодинамического равновесия энтропия замкнутой системы максимальна.\par
Пусть количество микросостояний, которые соответствуют текущим переменным состояния системы $\omega$, тогда
\[S = k \ln{\omega} \]
В адиабатном процессе энтропия не изменяется, т. к. $\delta Q = 0$\par
\law{Второе начало термодинамики (формулировка 1)}{
	Любая замкнутая система стремится к максимуму энтропии.
}
В открытых системах при некоторых условиях энтропия может уменьшаться.\par
\df{Вечный двигатель I рода}{гипотетическое устройство, нарушающее первое начало термодинамики.}
\df{Вечный двигатель II рода}{гипотетическое устройство, нарушающее второе начало термодинамики. \textbf{Пример:} давайте охладим мировой океан на $1\;\uK$, получим очень много энергии.}
\law{Второе начало термодинамики (формулировка 2)}{
	Процесс передачи энергии от горячего тела к холодному необратим.
}
\law{Второе начало термодинамики (формулировка Клаузиуса)}{
	Невозможно перевести тепло от более холодной системы к более горячей при отсутствии одновременных изменений в обеих системах или окружающих телах.
}
\law{Второе начало термодинамики (формулировка Кельвина)}{
	Невозможно осуществить такой периодический процесс, единственным результатом которого было бы получение работы за счёт теплоты, взятой от одного источника.
}



\subsection{Тепловые двигатели}
\begin{wrapfigure}{r}{0.5\textwidth}
	\def\svgwidth{250px}
	\input{images/carnot-cycle.pdf_tex}
\end{wrapfigure}
\df{Тепловой двигатель}{устройство, преобразущее $U$ в механическую работу. У тепловой машины должны быть рабочее тело, нагреватель и холодильник. Температура нагревателя обозначается $T_1$, а холодильника -- $T_2$. Тепловая машина должна работать циклически.}
\df{КПД теплового двигателя}{отношение совершеннной машиной работы к теплу, полученному от нагревателя.}
\[\eta=\frac{A}{Q_1}\]\par
\df{Идеальный тепловой двигатель}{двигатель, работающий по циклу Карно.}
\df{Цикл Карно}{цикл, состоящий из двух адиабат (4-1 и 2-3) и изотерм (1-2 и 3-4) (см. рис.).}

\ulaw{
	КПД теплового двигателя, работающего по циклу Карно, равен \[\eta=\frac{T_1 - T_2}{T_1}\]
}
\begin{proof}
	Рассмотрим открытую систему, состоящую из рабочего тела в двигателе. Её работа равна
	\[A=\oint{\delta A}=\oint{(\delta Q - \dlta U)}=\oint{T \dlta S} - \oint{\dlta U}=\oint{T \dlta S}\]
	Заметим, что изменение энтропии во время процессов 4-1 и 2-3 равно 0, и, очевидно, изменение термературы во время процессов 1-2 и 3-4 -- 0. Значит, график процесса в координатах TS -- прямоугольник, следовательно:
	\[A=\dlta T \cdot \dlta S=(T_1-T_2)\cdot\frac{Q_1}{T_1}
	\thus
	\eta=\frac{A}{Q_1}=\frac{(T_1-T_2)Q_1}{Q_1\cdot T_1}=\frac{T_1-T_2}{T_1}\]
\end{proof}

\df{Холодильный коэффициент}{отношение полученной от холодильника теплоты к работе, совершенной над рабочим телом.}
\[\epsilon = \frac{Q_2}{-A}\]
\ulaw{
	Для идеальной холодильной машины выполняется:
	\[\epsilon = \frac{T_2}{T_1-T_2}\]
}
\begin{proof}
	\[\epsilon = \frac{Q_2}{-A} = \frac{A-Q_1}{-A} = \frac{A-\frac{A}{\eta}}{-A} =
	\frac{1-\eta}{\eta} = \frac{\frac{T_2}{T_1}}{\frac{T_1-T_2}{T_1}} = \frac{T_2}{T_1-T_2} \]
\end{proof}

\law{Теорема Карно}{
	Максимальный КПД теплового двигателя равен
	\[\eta_{max}=\frac{T_1-T_2}{T_1}\]
}
\begin{proof}
TODO, в учебнике не очень понятно\par
\end{proof}



\subsection{Теплоёмкость}
\df{Теплоёмкость}{отношение переданного телу количества теплоты к изменению его температуры. [$\frac{\uJ}{\uK}$]}
\[C=\frac{\delta Q}{\dlta T}\]
\df{Удельная теплоёмкость}{теплоёмкость $1\;\ukg$ вещества. [$\frac{\uJ}{\uK \cdot \umol}$]}	
\df{Молярная теплоёмкость}{теплоёмкость $1\;\umol$ вещества. [$\frac{\uJ}{\uK \cdot \umol}$]}
\[c = \frac{C}{m} \text{;} \quad
C_\nu = \frac{C}{\nu} \text{;} \quad
C_\nu = cM\]
\df{Политропный процесс}{равновесный процесс с постоянной теплоёмкостью.}
Предельные случаи: адиабатный процесс ($\delta Q = 0$, $C = 0$) и изотермический процесс ($\dlta T = 0$, $C = +\infty$).\par


Запишем первое начало термодинамики:\par
\[\delta Q = \dlta U + \delta A\]
\[\frac{\delta Q}{\nu \dlta T} = \frac{\dlta U}{\nu \dlta T} + \frac{\delta A}{\nu \dlta T}\]
Получим первое начало термодинамики в дифференциальной форме:
\[C_\nu=\frac{\dlta U}{\nu \dlta T} + \frac{p \dlta V}{\nu \dlta T}\]

\ulaw{
	Молярная теплоёмкость в изохорном процессе равна
	\[C_V = \frac{i}{2} R \]
}
\begin{proof}
	\[\dlta V = 0 \thus C_V = \frac{dU}{\nu \dlta T} = \frac{\frac{i}{2}\nu R \dlta T}{\nu \dlta T} = \frac{i}{2}R \]
\end{proof}

\law{Соотношение Майера}{
	Молярная теплоёмкость в изобарном процессе равна
	\[C_p = C_V + R \]
}
\begin{proof}
	\[p \isconst \thus p\dlta V = \nu R \dlta T \thus C_p = \frac{\dlta U}{\nu \dlta T} + \frac{p \dlta V}{\nu \dlta T} = C_V + \frac{\nu R \dlta T}{\nu \dlta T} = C_V + R\]
\end{proof}

\textbf{Формула для ЕГЭ}: по определению внутренней энергии,
\[U = \frac{i}{2} \nu R T = \nu C_V T\]



\subsection{Уравнение политропного процесса}
\law{Уравнение политропного процесса}{
	\[pV^{\alpha} \isconst \text{, где } \alpha = \frac{C_p-C_\nu}{C_V-C_\nu}\]	
}
\begin{proof}
	\[ \dlta U = U_1 - U_0 \]
	\[ \nu R \dlta T = \nu R T_1 - \nu R T_0 = (p + \dlta p)(V + \dlta V) - pV = p \dlta V + V \dlta p \]
	TODO\par
\end{proof}

\law{Уравнение адиабатного процесса}{
	\[pV^{\lambda} \isconst \text{, где } \lambda = \frac{C_p}{C_V} \]
	$\lambda$ называют показателем адиабаты или коэффициентом Пуассна.
}
\begin{proof}
	\[Q = 0 \thus C_\nu = 0 \thus \lambda = \alpha = \frac{C_p - C_\nu}{C_V - C_\nu} = \frac{C_p}{C_V}\]
\end{proof}
В случае идеального газа $\displaystyle \lambda = \frac{C_p}{C_V} = \frac{\frac{i}{2}R}{\frac{i + 2}{2}R} = \frac{i + 2}{i}$.

\law{Работа газа при изотермическом процессе}{
	\[A = \nu R T \cdot \ln{\frac{V_2}{V_1}} \]
}
\begin{proof}
	\[A = \int \delta A = 
	\int_{V_1}^{V_2}{p \dlta V} = 
	\int_{V_1}^{V_2}{\frac{\nu R T}{V} \dlta V} =
	\nu R T \int_{V_1}^{V_2}{\frac{\dlta V}{V}} = 
	\nu R T (\ln{V_2} - \ln{V_1}) = \nu R T \cdot \ln{\frac{V_2}{V_1}} \]
\end{proof}