\section{Термодинамика}



\subsection{Работа газа}
\law{}{
	Для любого газа выполняется:
	\begin{equation}
	A=\int_{V_0}^{V_1}{p(V)\cdot \mathrm{d}V}
	\end{equation}
}

Геометрический смысл (4.1.1): работа газа равна площади под графиком в координатах $p$ от $V$.\par
Очевидно, что при сжатии газа работа газа отрицательна; при расширении - положительна.\par



\subsection{Первое начало термодинамики}
\df{Теплота}{энергия, которой могут обмениваться тела без совершения работы.}
\df{Адиабатический процесс}{процесс, проходящий без обмена теплом с окружающей средой.}
Быстропротекающие процессы и процессы, проходящие в теплоизолированных сосудах, как правило, адиабатические.\par
\law{Первое начало термодинамики}{
	Количество тепла, которое передаётся телу, расходуется им на изменение внутренней энергии и совершение работы.
	\begin{equation*}
	\delta Q = \Delta U + \delta A
	\end{equation*}
}



\subsection{Второе начало термодинамики}
\df{Обратимый процесс}{процесс перехода из начального состояния в конечное, если возможно вернуть систему хотя бы одним способом в исходное состояние, причём так, чтобы во всех остальных телах не произошло изменений.}
Процесс распределения части по всему объёму сосуда из его части необратим.\par
\df{Энтропия}{мера необратимости рассеяния энергии; функция состояния системы, равная отношению приобретённой (или потерянной) энергии к термодинамической температуре.}
В состоянии термодинамического равновесия энтропия замкнутой системы максимальна.\par
\law{Второе начало термодинамики (формулировка 1)}{
	Любая замкнутая система стремится к максимуму энтропии.
}
В открытых системах при некоторых условиях энтропия может уменьшаться.\par
\df{Вечный двигатель I рода}{гипотетическое устройство, нарушающее первое начало термодинамики.}
\df{Вечный двигатель II рода}{гипотетическое устройство, нарушающее второе начало термодинамики.}
\textbf{Пример:} давайте охладим мировой океан на $1^\circ \text{K}$, получим очень много энергии.
\law{Второе начало термодинамики (формулировка 2)}{
	Процесс передачи энергии от горячего тела к холодному необратим.
}
\df{Тепловой двигатель}{устройство, преобразущее $U$ в механическую работу. У тепловой машины должны быть рабочее тело, нагреватель и холодильник. Тепловая машина должна работать циклически.}
\df{Идеальный тепловой двигатель}{двигатель, работающий по циклу Карно.}
TODO: цикл Карно, схема теплового двигателя, КПД