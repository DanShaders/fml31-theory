\section{Термодинамика}



\subsection{Работа газа}
\begin{oframed}
Для любого газа выполняется:
\begin{equation}
A=\int_{V_0}^{V_1}{p(V)\cdot \mathrm{d}V}
\end{equation}
\end{oframed}

Геометрический смысл (4.1.1): работа газа равна площади под графиком в координатах $p$ от $V$.\par
Очевидно, что при сжатии газа работа газа отрицательна; при расширении - положительна.\par



\subsection{Первое начало термодинамики}
\textbf{Теплота} - энергия, которой могут обмениваться тела без совершения работы.\par
\textbf{Адиабатический процесс} - процесс, проходящий без обмена теплом с окружающей средой.\par
Быстропротекающие процессы и процессы, проходящие в теплоизолированных сосудах, как правило, адиабатические.\par
\begin{oframed}
\textbf{Первое начало термодинамики}\par
Количество тепла, которое передаётся телу, расходуется им на изменение внутренней энергии и совершение работы.
\begin{equation*}
\delta Q = \Delta U + \delta A
\end{equation*}
\end{oframed}