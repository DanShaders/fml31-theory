\section{Деформации твёрдого тела}


\df{Относительное удлинение}{отношение абсолютного удлинения к длине тела.}
\df{Коэффициент Пуассона}{отношение относительного удлинения по поперечной оси к относительному удлинению по продольной, взятое со знаком минус.}
\[ \varepsilon = \frac{\Delta l}{l} \text{; }\: \mu = -\frac{\varepsilon_y}{\varepsilon_x} \]
\df{Механическое напряжение}{отношение проекции силы на нормаль сечения к площади сечения. [$\uPa$]}
\df{Модуль Юнга}{отношение механического напряжения к относительному удлинению (свойство вещества). [$\uPa$]}
\[ \sigma = \frac{F}{S} = E \varepsilon \]
\df{Объёмная плотность энергии}{отношение потенциальной энергии упругой деформации к объёму тела. [$\uPa$]}
\[ W = \frac{E_\text{п}}{V} = \frac{k \Delta l^2}{2 V} = \frac{ES \Delta l^2}{2 l V} = \frac{E \varepsilon^2}{2} =
	\frac{\sigma^2}{2 E} \]
\df{Предел прочности}{максимальное механическое напряжение, которое способен выдержать материал, не разрушаясь.}
TODO: график напряженности от относительного удлинения.